\subsection{Influencia de $k_{ij}$ sobre $f$}\label{subsec:funcf}

Sin importar la estrategia de forrajeo tenemos que :
\begin{enumerate}[label=(\alph*)]
\item \begin{equation}
  \lim_{k_{ij} \to  0 }f(k_{ij}) = \lim_{k_{ij} \to 0 }\frac{g(k_{ij}) a}{1+k_{ij}^\phi} = 0 
\end{equation}
ya que $ g(k_{ij}) \to (k_{ij} \to 0)$ y $\frac{a}{1+k_{ij}^\phi} < a$  para todo $k_{ij}$

\item $f_{3D} > f_{2D} \iff k_{ij} > 1$ donde $f_{nD}$ representa al valor de $f$ para espacios de b\'usqueda \emph{n-dimensionales}
\end{enumerate}

A continuaci\'on se analizan propiedades de $f$ que a nivel \emph{cuantitativo} difieren entre las distintas estrategias de forrajeo

\subsubsection{Grazing}

\begin{prop}
  $f$ tiene un punto m\'aximo en $\mathbb{R}^+$ si y solo s\'i $\phi > (D_i - 1)p_d$ 
\end{prop}

\begin{proof}
\mbox{}\\
Sea $ h = (D_i-1)p_d$\\
Dado que $f$ es diferenciable y:
  \begin{equation}
    \begin{aligned}
      f'(k_{ij}) &= a \frac{h k_{ij}^{h-1} (1 + k_{ij}^\phi) -\phi k_{ij}^{\phi-1+h}}{(1+k_{ij}^{\phi})^2} \\
            &= a \frac{(h-\phi)k_{ij}^{h-1 + \phi}  + h k_{ij}^{h-1} }{(1+k_{ij}^{\phi})^2}\\
            &= a \frac{(h-\phi)k_{ij}^\phi + h }{(1+k_{ij}^{\phi})^2}\\
     \end{aligned}
  \end{equation}
Luego tenemos que $f'$ posee un \emph{cero} en $\mathbb{R}^+$ si y solo s\'i $ h< \phi$ y en caso afirmativo tenemos que $k^*$ tal que $f(k^*) = 0$ es $(\frac{h}{\phi-h})^{\frac{1}{\phi}}$ adem\'as:
\begin{equation}
  f'(x) :
  \begin{cases}
    < 0 &; k_{ij}  > k^* \\
    > 0 &; k_{ji} < k^*
  \end{cases}
\end{equation}

Lo que indica que $k^*$ es un punto m\'aximo.
\end{proof}

De la proposici\'on anterior tambi\'en vemos la dependencia de $k^*$ en $\phi$, teniendo $h$ fijo en nuestro caso se observa que si $\phi$ esta suficientemente cercano a $h$ , $k^*$ es extremadamente grande, sin embargo se observa un decaim\'iento muy r\'apido y para valores de $\phi \geq 2h$  , $k^*$ se encuentra pr\'oximo a 1(y en realidad converge a 1 para $\phi \to \infty$) , v\'ease figura \ref{fig:kmaxGrazing}. \\

\begin{figure}
\begin{center}
 \includegraphics[width=0.9\textwidth]{./Plots/kmaxGrazing.pdf}
 \caption[$k^*, Grazing$]{$k^*$ en funci\'on a $\phi$ donde se observa la divergencia para valores cercanos a $h$ y la convergencia a 1 para valores elevados de $\phi$, ({\hwplotB}) es para el caso de ambientes de b\'usqueda $3D$ y ({\hwplotR}) $2D$, $h_{3D}$ y $h_{2D}$ denotan los l\'imites inferiores para $\phi$ que permiten la existencia de $k^*$}
 \label{fig:kmaxGrazing} 
\end{center}
\end{figure}



En el caso que $\phi < h$ tenemos que $f$ es mon\'otona creciente. Ambos casos se grafican en la figura \ref{fig:f1Grazing}.

\begin{figure}
\begin{center}
 \includegraphics[width=0.9\textwidth]{./Plots/f1Grazing.pdf}
 \caption[$f_1, Grazing$]{$f$ en funci\'on a $k_{ij}$, con $a =1$, en el panel de la izquierda tenemos $b = 0.1$ y en el de la derecha $b=1.$ , ({\hwplotB}) es para el caso de ambientes de b\'usqueda $3D$ y ({\hwplotR}) $2D$}
 \label{fig:f1Grazing} 
\end{center}
\end{figure}


\subsubsection{Sit}

Se tiene cualitativamente las m\'ismas caracter\'isticas que en el caso anterior, con la diferencia que en este caso $ h:= p_v + 2(D_i -1) p_d$, y por ende para $\phi \in ]2(D_i - 1) p_d , p_V + 2(D_i -1)p_d[$ tendr\'iamos un comportamiento mon\'ontono para $f_{sit}$ y la existencia de un m\'aximo para $f_{grazing}$.\\

La figura \ref{fig:f1Sit} muestra las semejanzas con el caso anterior, salvo la diferencia que en este caso $k^*>1$ para el caso $3D$ y adem\'as alcanza un valor m\'as alto que el caso $2D$.\\
En general tenemos que $ k^*_{Sit} > k^*_{grazing}$ .

\begin{figure}
\begin{center}
 \includegraphics[width=0.9\textwidth]{./Plots/f1Sit.pdf}
 \caption[$f_1, Sit$]{$f$ en funci\'on a $k_{ij}$, con $a =1$, para el caso de una estrategia de forrajeo \emph{Sit-and-Wait}, las dem\'as especificaciones se comparten con la figura \ref{fig:f1Grazing}}
 \label{fig:f1Sit} 
\end{center}
\end{figure}


\subsubsection{Active}

En este caso tenemos que :
\begin{equation}
  \lim_{k_{ij} \to 0 } f_{active}(k_{ij}) - f_{grazing}(k_{ij}) = 0 \ \ \land \lim_{k_{ij} \to \infty}f_{active}(k_{ij}) - f_{sit}(k_{ij}) = 0
\end{equation}

Por lo tanto en ambos extremos podemos esperar un comportamiento similar al descrito en los dos casos anteriores, esto es para $\phi > p_v + 2(D_i -1) p_d$ tenemos que $f$ decae exponencialmente a partir de un valor de $k_{ij}$ \emph{suficientemente grande}, y a su vez crece exponencialmente para valores \emph{suficientemente peque\~nos}, dado que $f \in C^1$ esto a su vez nos dice que debe existir un punto m\'aximo para $f$ sin embargo en este caso no tenemos una expresi\'on an\'alitica para $k^*$ salvo que cumple la siguiente relaci\'on:

\begin{equation}
  (k^*)^{\phi}((k^*)^{2p_v}(p_v+ h -\phi) + (h -\phi) + (k^*)^{2p_v - \phi}(p_v + h ) ) + h = 0
\end{equation}

Con $h$ igual que en el caso $grazing$.\\
De esta relaci\'on podemos obtener que :
\begin{equation}
  k^*_{Active} \in ] k^*_{Grazing} , k^*_{Sit} [
\end{equation}
Los detalles se dejan al lector como ejercicio.\\

A su vez para $\phi \leq 2(D_i - 1)p_d $ podemos esperar un crecimiento mon\'otono por parte de $f$. 

\begin{figure}
\begin{center}
 \includegraphics[width=0.9\textwidth]{./Plots/f1Active.pdf}
 \caption[$f_1, Active$]{$f$ en funci\'on a $k_{ij}$, con $a =1$, para el caso de una estrategia de forrajeo \emph{activa}, las dem\'as especificaciones se comparten con la figura \ref{fig:f1Grazing}}
 \label{fig:f1Active} 
\end{center}
\end{figure}



