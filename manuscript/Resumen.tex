\section{Resumen}
Dada su conexi\'on con la tasa de metabolismo y movimiento, la masa corporal de las especies es un caracter clave que determina gran parte de la funcionalidad de la especie, en este trabajo extendemos esta l\'inea de pensamiento al explorar como la masa coporal de las especies que componen el modulo de depredaci\'on intragruemial influencia la m\'axima posici\'on tr\'ofica observada en el sistema, su camino de estructuraci\'on (secuencia de invasiones \emph{plausibles}) y la zona de coexistencia. Derivamos criterios de coexistencia e invasibilidad en funci\'on de la masa de las especies interactuantes, encontramos que la forma de estos criterios es independiente del nivel de productividad basal, dimensi\'on del espacio de b\'usqueda del depredador y estrategia de forrajeo; los cu\'ales sin embargo tienen una influencia cuantitativa, m\'as a\'un los dos \'ultimos influencian el comportamiento cualitativo de partes de la relaci\'on con respecto a cambios en la raz\'on de masas depredador presa pesentes en el m\'odulo. Debido a la influencia que ejercen sobre estos procesos la combinaci\'on de masas afecta la longitud de la cadena tr\'ofica presente en el m\'odulo. Estos resultados sugieren prometedoras conexiones entre el proceso de ensamblaje de una comunidad y la masa corporal de las especies pesentes en el pool regional de especies y la comunidad receptora.


\palabrasclave{Masa coporal, metabolismo, depredaci\'on intragremial, ensamblaje, cadena tr\'ofica}
