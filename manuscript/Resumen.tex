\section{Resumen}
Dada su relaci\'on con la tasa de metabolismo y movimiento, la masa corporal de las especies es una variable clave que describe bien aspectos funcionales de las especies. En este trabajo se exploramos la relaci\'on entre la masa coporal de las especies que componen el modulo de depredaci\'on intragremial y la m\'axima posici\'on tr\'ofica observada en el sistema, su camino de estructuraci\'on (secuencia de invasiones \emph{plausibles}) y la zona de coexistencia. Derivamos criterios de coexistencia e invasibilidad en funci\'on de la masa de las especies interactuantes. Encontramos que la forma de estos criterios es independiente del nivel de productividad basal, dimensi\'on del espacio de b\'usqueda del depredador y estrategia de forrajeo; los cu\'ales tienen una influencia cuantitativa. M\'as a\'un la dimensi\'on del espacio de b\'usqueda y la estrategia de forrajeo influencian el comportamiento cualitativo de partes de la relaci\'on con respecto a cambios en la raz\'on de masas depredador presa pesentes en el m\'odulo. Debido a la influencia que ejercen sobre estos procesos la masa de las especies afecta la longitud de la cadena tr\'ofica presente en el m\'odulo. Estos resultados sugieren relaciones entre el proceso de ensamblaje de una comunidad y la masa corporal de las especies presentes en el conjunto de potenciales colonizadors y la comunidad receptora.


\palabrasclave{Masa coporal, metabolismo, depredaci\'on intragremial, ensamblaje, cadena tr\'ofica}
