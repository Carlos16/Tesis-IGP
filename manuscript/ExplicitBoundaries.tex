\subsection{C\'alculo Expl\'icito de zonas de invasibilidad}

En esta secci\'on derivamos los criterios de invasibilidad en funci\'on de la raz\'on de masas ($k_{\RC},k_{\CP}$ y $k_{\RP}$) y la masa del depredador tope $m_P$, asumiendo que:
\begin{enumerate}
\item $D_R = D_C = D$
\item $\beta_R = \beta_C = \beta_P = \beta$
\end{enumerate}


\subsubsection{C $\to$ R}

\begin{equation}
  \frac{dC}{dt} >0 \iff  m_P > \zeta_1(k_{\RC},k_{\CP}) = k_{\CP}^{-1}(\frac{\chi_1}{\chi_0})^{\frac{1}{h_R +1 - 2\beta }}
\end{equation}
Donde:
\begin{equation}
  \begin{aligned}
    h_R &= p_v + 2(D_R - 1)p_d \\
    \chi_0 &= \varepsilon_1 \kappa_0\alpha_{0,1} f_1(k_{\RC})k_{\RC}^{1-\beta} \\
    \chi_1 &= q_{0,1} 
  \end{aligned}
\end{equation}
Entonces:
\begin{equation}
\mathbf{Z(I_{\C\to \R})} := \{ (k_{\RC},k_{\CP},m_P) \in \mathbb{R}^3_+ / m_p > \zeta_1(k_{\RC},k_{\CP}) \}
\end{equation}

\subsubsection{P $\to$ R}

\begin{equation}
  \frac{dP}{dt} >0 \iff  m_P > \zeta_2(k_{\RC},k_{\CP}) = (\frac{q_{0,2}}{\eta_0})^{\frac{1}{h_R +1 - 2\beta}}
\end{equation}
Donde:
\begin{equation}
  \begin{aligned}
    \eta_0 &= \varepsilon_2 \kappa_0\alpha_{0,2} f_2(k_{\RP})k_{\RP}^{1-\beta} \\
  \end{aligned}
\end{equation}


\begin{equation}
\mathbf{Z(I_{\PP \to \R})} := \{ (k_{\RC},k_{\CP},m_P) \in \mathbb{R}^3_+ / m_p > \zeta_2(k_{\RC},k_{\CP}) \}
\end{equation}


Definamos:
\begin{equation}
  h = p_v + 2(D-1)p_d
\end{equation}

\subsubsection{P $\to$ C-R}
\begin{equation}
  \frac{dP}{dt}  >0 \iff m_P > \zeta_3(k_{\RC},k_{\CP}) = (\frac{\xi_4\xi_2}{\xi_3 + \xi_4 - q_{0,2}})^{\frac{1}{h + 1 - 2\beta}}
\end{equation}

Donde:
\begin{equation}
  \begin{aligned}
    \xi_0 &= \frac{q_{0,1} k_{\CP}^{\beta - h}}{\epsilon_1 \alpha_{0,1} f_1(k_{\RC})}\\
    \xi_1 &= \frac{r_0 k_{\RC}^{\beta -1}k_{\CP}^{\beta-h}}{\alpha_{0,1} f_1(k_{\RC})}\\
    \xi_2 &= \frac{\xi_0}{k_0 k_{\RP}^{1-\beta}} \\
    \xi_3 &= \epsilon_2 \alpha_{0,2} f_2(k_{\RP}) \xi_0\\
    \xi_4 &=\epsilon_3 \alpha_{0,3} f_3(k_{\CP}) \xi_1
  \end{aligned}
\end{equation}


Por lo tanto 
\begin{equation}
\mathbf{Z(I_{\PP \to \C-\R})} := \{ (k_{\RC},k_{\CP},m_P) \in \mathbb{R}^3_+ / m_p > \max\{\zeta_1(k_{\RC},k_{\CP}),\zeta_3(k_{\RC},k_{\CP})\} \}
\end{equation}
\subsubsection{C $\to$ P-R}
\begin{equation}
  \frac{dC}{dt}  >0 \iff m_P < \zeta_4(k_{\RC},k_{\CP}) = (\frac{\gamma_4\gamma_2}{\gamma_5 + \gamma_4 - \gamma_3})^{\frac{1}{h + 1 - 2\beta}}
\end{equation}

Donde:
\begin{equation}
  \begin{aligned}
    \gamma_0 &= \frac{q_{0,2}}{\varepsilon_2 \alpha_{0,2} f_2(k_{\RP})}\\
    \gamma_1 &= \frac{r_0 k_{\RP}^{\beta -1}}{\alpha_{0,2} f_2(k_{\RC})}\\
    \gamma_2 &= \frac{\gamma_0}{\kappa_0 k_{\RP}^{1-\beta}} \\
    \gamma_3 &= \varepsilon_1\alpha_{0,1} f_1(k_{\RP}) k_{\CP}^{h -1} \gamma_0\\
    \gamma_4 &=\alpha_{0,3} f_3(k_{\CP}) \gamma_1 \\
    \gamma_5 &= q_{0,1} k_{\CP}^{\beta-1}
  \end{aligned}
\end{equation}

Por lo tanto tenemos que:

\begin{equation}
\mathbf{Z(I_{\C \to \PP-\R})} := \{ (k_{\RC},k_{\CP},m_P) \in \mathbb{R}^3_+ / \zeta_4(k_{\RC},k_{\CP}) > m_p > \zeta_2(k_{\RC},k_{\CP}) \}
\end{equation}

De donde la region de invasibilidad mutua $Z_{IM} := Z(I_{\C \to \PP-\R}) \cap Z(I_{\PP \to \C-\R})$ , resulta:


\begin{equation}
\mathbf{Z_{IM}} := \{ (k_{\RC},k_{\CP},m_P) \in \mathbb{R}^3_+ / \zeta_4(k_{\RC},k_{\CP}) > m_p > \max \{ \zeta_2(k_{\RC},k_{\CP}) , \zeta_1(k_{\RC},k_{\CP}) , \zeta_3(k_{\RC},k_{\CP}) \}
\end{equation}

Adem\'as tenemos que la regi\'on de $k_{\RC} \times k_{\CP} \times m_P$ asociada a $E_1$ denotada por $R(E_1)$ esta determinada por
\begin{equation}
  R(E_1) := \{ (k_{\RC},k_{\CP},m_P) \in \mathbb{R}^3_+ / \zeta_4(k_{\RC},k_{\CP}) > m_p > \zeta_3(k_{\RC},k_{\CP}) \}
\end{equation}

Definimos:
\begin{equation}
  \Delta_\varepsilon = \varepsilon_1 \varepsilon_3 - \varepsilon_2
\end{equation}
Para $\Delta_\varepsilon <0$ tenemos que la regi\'on para $D >0$ esta definida por:
\begin{equation}
    D >0  \iff  m_P < (\frac{\gamma_0}{\gamma_1})^{\frac{1}{2h_R - h_C + 1 - 2 \beta}}
\end{equation}
Donde:
\begin{equation}
  \begin{aligned}
    \gamma_ 0 &= \varepsilon_3 r_0\alpha_{0,3}k_{\RP}^{\beta-1}f_3(k_{\CP})\\
    \gamma_1 &= -\Delta_\varepsilon \kappa_0 \alpha_{0,1}\alpha_{0,2}f_1(k_{\RC})f_2(k_{\RP}) k_{\CP}^{h_R - 1}    
  \end{aligned}
\end{equation}
Notamos que el exponente se simplifica al encontrado en casos anteriores para $h_R  = h_C$.



