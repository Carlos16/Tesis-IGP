\section{RESULTADOS}

\subsection{Invasibilidad}

Para una zona $Z$ en particular definimos la \emph{zona relativa} al tipo de comunidad $n$ como $Z_n = Z \cap C_n$. \\
A continuaci\'on se describen resultados para la combinaci\'on de estrategias de forrajeo \emph{Grazing-Grazing-Active}. Dado que en los otros dos casos se obtiene una respuesta cualitativa similar estos no ser\'an detallados.

\subsubsection{$ P \to R-C$}

En todos los casos explorados $Z(IC_4)$ tiende a crecer con respecto a aumentos en la masa del depredador $m_P$, y adem\'as forman una secuencia encajante. Dicho crecimiento es en cierta manera similar al que experimenta una zona para una valor de $m_P$ en particular, respecto a aumentos en la productividad basal $k_0$. Sin embargo el crecimiento no es uniforme, las zonas relativas si bien tienden a crecer lo hacen de manera desigual dependiendo adem\'as del valor del par\'ametro $\phi$.\\

El valor de $\phi$ controla en cierta manera la forma de $Z(IC_4)$, observandose por lo general una relaci\'on inversa del valor de $phi$ y el \'area total de $Z(IC_4)$ irrespectivamente de la masa. Sin embargo afecta de manera desigual los distintos tipos de Comunidades.\\ Por ejemplo, en el caso de comunidades con distribuci\'on de masa $C_1$ el par\'ametro es pr\'acticamente irrelavante sobre la zona relativa $Z(IC_4)_1$, y por el contrario para comunidades $C_5$ el cambio es dram\'atico, dado que entre $\phi  = 0.02 - 2$, $Z(IC_4)_5$ pasa de ser no vac\'ia y cubrir en su totalidad(en el caso 3D) a $C_5$ (i.e $Z(IC_4)_5 = C_5$) para todas las masas $m_P$ exploradas, a ser vac\'io para todo $m_P$ en el segundo caso.\\
La dimensi\'on del espacio de b\'usqueda afecta el \'area total y las \'areas relativas de $Z(IC_4)$.\\
En el primer caso observamos que generalmente es mayor en espacios tridimensionales(3D) y esta diferencia se acent\'ua conforme la masa $m_P$ aumenta.\\
En el segundo caso tenemos un patr\'on muy similar pero en este caso para $m_P$ bajos se pueden observar comunidades donde el \'area de la zona relativa es mayor en ambientes bidimensionales, como por ejemplo $Z(IC_3)$ para $m_P = 10^{-10}$. \\
Adem\'as tenemos que la distribuci\'on del \'area de las zonas \emph{relativas} es diferente en espacios 3D y 2D.\\
Algo que resaltar es que el borde inferior(en escala logar\'itmica) tiende a ser \emph{paralelo} a $\log_{10}(K_{CP}) = -log_{10}(K_{RC})$ en el caso 3D y no en el caso 2D.


\begin{figure}
  \centering
  \includegraphics[width = 0.9\textwidth]{./Plots/ZIC4b2e0.pdf}
  \caption[Env $Z(IC4)$]{\emph{Envolturas de Invasibilidad} para el caso de el depredador tope $P$ como invasor frente a una comunidad receptora formada por $R-C$. La fila superior es para espacios de b\'usqueda bidimensionales y la inferior tridimensionales, las columnas de izquierda a derecha aumentan el nivel de productividad basal $k_0$, siendo $0.01,0.1,1$ y $3,30,300$ en la primera y segunda fila respectivamente.Las diferentes lineas implican distintas masas de depredador $m_P$ :({\hwplotR}) $10^5 kg$,  ({\hwplotY}) $1kg$, ({\hwplotG}) $10^{-5}kg$ y ({\hwplotB}) para $10^{-10}kg$. ({\hwplotK}) separa las zonas donde $K_{RC},K_{CP},k_{RP}$ son mayores o menores que 1 respectivamente.}
  \label{fig:Z(IC4)}
\end{figure}

\subsubsection{$C \to R-C$}
De forma similar al caso anterior tenemos que el \'area total de $Z(IC_5)$ esta relacionada positivamente con el valor de $m_P$, sin embargo a diferencia del caso anterior las zonas para distintos $m_P$ si bien tienen intersecci\'on no vac\'ia ,no forman una secuencia encajante. Es decir a parte de crecimiento tenemos un desplazamiento conforme aumenta $m_P$. Igual que en el caso anterior este patr\'on se asemeja al observado para un $m_P$ en particular respecto a la variaci\'on de $k_0$. \\
En este caso es bastante notorio la no uniformidad en el \'area de las zonas relativas.\\

El par\'ametro $\phi$ juega un papel importante en la topolog\'ia de $Z(IC_5)$, para valores peque\~nos el conjunto es conexo, sin embargo para $\phi = 2$ el conjunto tiene 2 componentes, caracterizadas por la distribuci\'on de sus \'areas relativas. La distancia entre estas componenes aumenta con $m_P$ y $k_0$ , para $m_P$($k_0$) elevados podemos distinguirlas por el hecho que una de ellas $\vartheta_C^1$intersecta principalmente las zonas $C_4,C_5,C_6$ y la otra $\vartheta_C^2$ las zonas $C_1,C_2$, siendo la primera la que presenta mayor \'area. \\
Para $\phi = 0.2 ,0.02$  tenemos que $Z(IC_5)$ intersecta principalmente a las zonas $C_2,C_3,C_4$ y posee una \emph{cola} que es paralela(en escala logar\'itmica) a $\log_{10}(K_{CP}) = -log_{10}(K_{RC})$ y que dependiendo del valor de $m_P$($k_0$) esta contenido en las zonas $C_1,C_2,C_6$ o $C_3,C_4,C_5$, es decir $k_{RP} > 1$ o $k_{RP}<1$. \\

La dimensi\'on del espacio de b\'usqueda igual que en el caso anterior afecta el \'area total de $Z(IC_5)$ la cual es mayor en ambientes tridimensionales. Sin embaro la distribuci\'on de las zonas \emph{relativas} es similar en los dos casos. \\
Para $\phi = 2$ se observa que el \'area de $\vartheta_C^2$ es por lo general mayor en ambientes bidimensionales y lo contrario ocurre con el otro componente.

\begin{figure}
  \centering
  \includegraphics[width = 0.9\textwidth]{./Plots/ZIC5b2e0.pdf}
  \caption[Env $Z(IC5)$]{\emph{Envolturas de Invasibilidad} para el caso de el depredador intermedio $C$ como invasor frente a una comunidad receptora formada por $R-P$. Las dem\'as especificaciones se comparten con la figura ~\ref{fig:Z(IC4)}.}
\end{figure}





















