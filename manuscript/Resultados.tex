\section{RESULTADOS}

\subsection{Mecanismos de Variaci\'on}

\subsubsection{Adici\'on}
\myparagraph{$k_{\CP} = k_{\RC}$}
Irrespectivamente de la combinaci\'on de estrategias de forrajeo, la dimensi\'on en la que se mueve la presa y el nivel de productividad basal $K_0$, se observa un aumento en el rango de raz\'on de masas presa-depredador $k_{\CP}$ asociados con el cumplimiento de los criterios de invasibilidad $I_{C1},I_{C2},I_{C3},I_{C4}$ al aumentar la masa del depredador tope $m_\PP$, por lo que la expresi\'on del mecanismo de adicci\'on tiene una relaci'on positiva con la masa del depredador tope,sin embargo dicha relaci\'on es mucho m\'as fuerte en ambientes 3D, adem\'as la forma espec\'ifica de los bordes de los distintos criterios si son afectadas por estas caracter\'isticas del sistema y por el valor del par\'ametro $phi$ .

\paragraph*{Combinaci\'on 1(Captura Activa-Pastoreo-Pastoreo)} El borde inferior de $Z(I_{C2})$ disminuye conforme aumenta $m_P$ , a valores peque\~nos $R(I_{C2})$ es mayor en h\'abitats 2D sin embargo lo opuesto se da a valores altos de $m_P$, este borde tambi\'en disminuye con el incremento de la productividad basal, un patr\'on similar se observa en relaci\'on a $I_{C3}$. \\
Respecto a $I_{C4}$, el valor del par\'ametro $phi$ afecta el borde superior de $Z(I_{C4})$, valores peque\~nos(e.g $\phi = 0.02$),dan un patr\'on similar al descrito para $I_{C2}$.\\
Para valores elevados($\phi = 2$), $Z(I_{C4})$ se divide en 2 zonas , 1 y 2 donde los valores de $k_{cp}$ presentes en la zona 2 son mayores a los de la zona 1. El borde inferior de la zona 1 se comporta de manera similar a lo descrito para $I_{C2}$ y es bastante cercano al borde inferior de $I_{C2}$, y el borde superior tambi\'en aumenta con respecto a $m_P$, sin embargo en este caso para ambientes 3D se observa que hay un valor de $m_P$( el cual disminuye con respecto $K_0$) debajo del cual el $k_{cp}$ asociado con $I_{C4}$ solo pertenece a la zona 2, cosa que no se observa en ambientes 2D. El borde inferior de la zona 2 es pr\'acticamente constante en ambos ambientes, sin embargo el valor es mucho menor en ambientes 2D,en ambos casos es dependiente de la productividad basal.

\paragraph*{Combinaci\'on 2(Captura Activa-Captura Activa-Captura Activa)}
El patr\'on que se observa con respecto a $I_{C2},I_{C3},I_{C4}$  para valores peque\~nos de $\phi$ es similar al descrito anteriormente, con la ligera diferencia de que en el caso 3D, existe un valor de $m_P$,debajo del cual el $R(I_{C3})$ es mayor que $R(I_{C2})$ y $R(I_{C4})$, este valor disminuye con respecto al aumento de la productividad basal.\\
En el caso que $\phi=2$, $Z(I_{C2}),Z(I_{C3}),Z(I_{C4})$ presentan un borde inferior que se comporta de igual manera a lo descrito previamente, y adem\'as poseen un borde superior que crece con respecto a $m_\PP$ y $K_0$, en ambientes 2D, los bordes de $Z(I_{C2})$ y $Z(I_{C4})$ son bastante cercanos, sin embargo en ambientes 3D existe un valor de $m_\PP$ a partir del cual el borde superior de $Z(I_{C2})$ se hace mucho mayor que el de $Z(I_{C4})$,  y un valor de $m_\PP$ a partir del cual el borde superior de $Z(I_{C3})$ es mayor al borde superior de $Z(I_{C4})$.

\paragraph*{Combinaci\'on 3(Captura Pasiva-Captura Pasiva- Captura Activa)}
Salvo peque\~nas diferencias cuantitativas, el patr\'on es similar al descrito para la combinaci\'on 2.

\myparagraph{$k_{\CP} \not= k_{\RC}$}
De forma similar al caso anterior, el \'area total de $Z(I_{C2}),Z(I_{C3}),Z(I_{C4})$ aumenta conforme aumenta la masa $m_\PP$, para todas las combinaciones de estrategias de forrajeo, dimensi\'on de habitat de recurso, nivel de productividad basal y $phi$. Por lo que la expresi\'on del mecanismo de adici\'on tiene igual relaci\'on con la masa $m_\PP$ al del caso $k_{\CP}= k_{\RC}$, observ\'andose nuevamente un cambio mucho mayor con respecto a $m_\PP$ en ambientes 3D y que las formas espec\'ificas de cada zonas $Z$ dependen de la combinaci\'on de par\'ametros usada.

\paragraph*{Combinaci\'on 1(Captura Activa-Pastoreo-Pastoreo)}Para un valor de particular de $m_\PP$, el borde inferior de $Z(I_{C2})$ disminuye conforme aumenta $k_{\CP}$.De forma similar al caso anterior el borde disminuye con respecto a aumentos en $K_0$ y $R(I_{C2})$ es mayor en ambientes 2D para valores peque\~nos de $k_{\RC}$ y menor para altos valores.A su vez conforme $m_\PP$ aumenta este borde disminuye, para valores peque\~nos de $m_\PP$,el \'area de $Z(I_{C2})$ es mayor en ambientes 2D, y mayor en ambientes 3D para valores altos de $m_\PP$ .Un patr\'on semejante se observa para $I_{C3}$ \\ Una diferencia entre ambientes 2D y 3D es que en el primer caso existe un valor de $k_{\RC}$,para todos los $m_\PP$ explorados, a partir del cual $R(I_{C3})$ es mayor a $R(I_{C2})$, dicho $k_{\CP}$ aumenta con respecto a $K_0$; en ambientes 3D solo se observa para $m_\PP = 10^{-10} ,10^{-5}$.\\
Con respecto a $I_{C4}$ , la forma de $Z(I_{C4})$ para un valor particular de $m_\PP$ var\'ia con $\phi$. Para $\phi = 0.02$  existe un borde superior que aumenta mon\'otonamente con respecto a $k_{\CP}$. Para $\phi = 2$ la variaci\'on en el borde superior no es mon\'otona, existe un valor $k_{\CP}$ a partir del cual cual el borde empieza a decrecer hasta un punto en el cual crece abruptamente(excepto para $m_\PP = 10^{-10}$. La diferencia entre ambientes 2D y 3D es que el decrecimiento es mayor en el segundo caso.En ambientes 3D para $m_\PP = 10^{-10}$ y $K_0 = 3,30$ $Z(I_{C4})$ se divide en 2 Zonas y entre ellas existe un rango de valores de $k_{\CP}$ para el cual ning\'un $k_{\CP}$ pertence a $Z(I_{C4})$. \\
En ambos casos vez existe un valor de $k_{\CP}$ debajo del cual ning\'un valor de $k_{\RC}$ dentro del espacio explorado pertenece a $Z(I_{C4})$,dicho valor es menor para ambientes 2D para bajos $m_\PP$ y menor en ambientes 3D para valores altos de $m_\PP$. A su vez este valor decrese conforme aumenta $K_0$.

\paragraph*{Combinaci\'on 2(Captura Activa-Captura Activa-Captura Activa)}
Para $\phi  = 0.02$ el patr\'on es similar al descrito para la combinaci\'on anterior. \\
Para $\phi = 2$, $Z(I_{C2})$ presenta un borde inferior que se comporta de forma silimar al descrito anteriormente, y un borde superior que crece con respecto a $k_{\CP}$, $m_\PP$ y $K_0$. El borde inferior de $Z(I_{C3})$ se comporta de forma similiar, sin embargo el borde superior decrece con $k_{\CP}$. Igual que en la combinaci\'on anterior se observa la presencia de un $k_{\CP}$ a partir del cual el borde inferior de $Z(I_{C3})$ es menor al de $Z(I_{C2})$, el cual es afectado por $K_0$ y es observado con mayor frecuencia en ambientes 2D. En el caso de $Z(I_{C4})$ su forma es cualitativamente diferente en ambientes 2D y 3D, en ambientes 2D ,para todos los $m_\PP$ explorados, existe un $k_{\CP}$ debajo del cual existen un rango de valores de $k_{\RC}$ entre el borde inferior y superior de $Z(I_{C4})$ que no pertenecen a \'el, dicho $k_{\CP}$ no se observa en ambientes 3D. Para ambos ambientes existe un rango de valores de $k_{\CP}$ entre los cuales se tiene que el conjunto de valores de $k_{\RC}$ que pertencen a $Z(I_{C4})$ no es conexo,la posici\'on de dicho rango var\'ia con $m_\PP$ y con $K_0$. Adem\'as en ambos casos el borde superior crece con respecto a $k_{\RC}$.

\paragraph*{Combinaci\'on 3(Captura Pasiva-Captura Pasiva- Captura Activa)}
Para $\phi = 0.02$ el patr\'on se asemeja al descrito para la combinaci\'on 2.
Para $\phi = 2$ en ambientes 2D el patr\'on es semejante al descrito en la combinaci\'on 2 . En el caso de ambientes 3D existe una diferencia respecto a la combinaci\'on 2, en este caso para $Z(I_{C4})$ y valores de $m_\PP = 10^{-5},1,10^{5}$ la forma no es cualitativamente diferente a la que se obtiene en ambientes 2D, es decir existe un valor de $k_{\CP}$ debajo del cual el conjunto de valores de $k_{\RC}$ que pertencen a $Z(I_{C4})$ no es conexo.


\paragraph*{Combinaci\'on 2(Captura Activa-Captura Activa-Captura Activa)}
Para $\phi  = 0.02$ el patr\'on es similar al descrito para la combinaci\'on anterior. \\
Para $\phi = 2$, $Z(I_{C2})$ presenta un borde inferior que se comporta de forma silimar al descrito anteriormente, y un borde superior que crece con respecto a $k_{\RC}$, $m_\PP$ y $K_0$. El borde inferior de $Z(I_{C3})$ se comporta de forma similiar, sin embargo el borde superior decrece con $k_{\CP}$. Igual que en la combinaci\'on anterior se observa la presencia de un $k_{\CP}$ a partir del cual el borde inferior de $Z(I_{C3})$ es menor al de $Z(I_{C2})$, el cual es afectado por $K_0$ y es observado con mayor frecuencia en ambientes 2D. En el caso de $Z(I_{C4})$ su forma es cualitativamente diferente en ambientes 2D y 3D, en ambientes 2D ,para todos los $m_\PP$ explorados, existe un $k_{\CP}$ debajo del cual existen un rango de valores de $k_{\RC}$ entre el borde inferior y superior de $Z(I_{C4})$ que no pertenecen a \'el, dicho $k_{\CP}$ no se observa en ambientes 3D. Para ambos ambientes existe un rango de valores de $k_{\CP}$ entre los cuales se tiene que el conjunto de valores de $k_{\RC}$ que pertencen a $Z(I_{C4})$ no es conexo,la posici\'on de dicho rango var\'ia con $m_\PP$ y con $K_0$. Adem\'as en ambos casos el borde superior crece con respecto a $k_{\RC}$.

\paragraph*{Combinaci\'on 3(Captura Pasiva-Captura Pasiva- Captura Activa)}
Para $\phi = 0.02$ el patr\'on se asemeja al descrito para la combinaci\'on 2.
Para $\phi = 2$ en ambientes 2D el patr\'on es semejante al descrito en la combinaci\'on 2 . En el caso de ambientes 3D existe una diferencia respecto a la combinaci\'on 2, en este caso para $Z(I_{C4})$ y valores de $m_\PP = 10^{-5},1,10^{5}$ la forma no es cualitativamente diferente a la que se obtiene en ambientes 2D, es decir existe un valor de $k_{\CP}$ debajo del cual el conjunto de valores de $k_{\RC}$ que pertencen a $Z(I_{C4})$ no es conexo.


\subsubsection{Inserci\'on}
\myparagraph{$k_{\CP} = k_{\RC}$}


\myparagraph{$k_{\CP} \not= k_{\RC}$}

\subsubsection{Omnivorismo}

\myparagraph{$k_{\CP} = k_{\RC}$}


\myparagraph{$k_{\CP} \not= k_{\RC}$}


\subsection{Posici\'on Tr\'ofica}

\subsubsection{$K_{\CP} = K_{\RC}$}


\subsubsection{$K_{\CP} \not= K_{\RC}$}













