\section{DISCUSION}


Nuestros resultados muestran que dependiendo de la combinanci\'on de masas presente en el m\'odulo IGP el curso del ensamblaje del m\'odulo puede tomar distintos caminos, en particular se observa que ciertas combinaciones restringen la \emph{adici\'on} de depredadores(tope o intermedio) o la \emph{inserci\'on} del depredador intermedio restringiendo de esta manera posibles cambios en la FCL observada en la comunidad. Estas observaciones complementan el trabajo realizado por \citet{holt2002food} en el cual sugiri\'o el hecho que procesos espaciales como \emph{colonizaci\'on} o \emph{extinci\'on} ser\'ian limitantes de FCL, para ser m\'as precisos estos procesos estar\'ian modulando la estructura de la \emph{red tr\'ofica} presente en la comunidad y a su vez la estructura en un instante determinado limita colonizaciones o extinciones futuras\citep{pawar2009community,holt2002}.  Los resultados obtenidos en este trabajo sugieren que esta \emph{feed-back} entre la estructura de la red y el procesos de ensamblaje estar\'ia a su vez influenciado por la distribuci\'on de masas presentes tanto en la comunidad receptora como en el pool de colonizadores.\\

Dada la influencia que tiene la masa corporal de las especies sobre la tasa de metabolismo(citar), y su subsecuente influencia sobre para\'ametros tanto \emph{intra}($r,K,q_0$)  e \emph{inter-poblacionales} ($\alpha_i$) es plausible pensar que la distribuci\'on de masas tendr\'ia un efecto sobre los mecanismos de invasibilidad y las zonas de coexistencia del m\'odulo IGP , irrespectivamente de las particularidades del modelo usado(e.g respuesta funcional Tipo I). \\

La similaridad en la variaci\'on en el \'area total de todas las zonas con respecto a aumentos en $m_P$ y $k_0$ estar\'ia indicando que el mayor efecto que tiene $m_P$ sobre la invasibilidad es el hecho de aumentar la capacidad de carga de $R$ y esto se debe al hecho de que si bien $m_P$ afecta el valor de diversos par\'ametros del modelo(e.g tasas de mortalidad $q$) las constantes que acompa\~nan a dichos par\'ametros son relativamente peque\~nas comparadas con $k_0$.\\
En nuestros resultados $m_P$ cumple el papel de reescalar los valores posibles de masas presentes en la comunidad, es decir $m_P$ mayores implican un mayor valor de $m_R$ y $m_C$ para una determinada combinaci\'on de $k_{\RC}$ y $k_{\CP}$. Para un $m_P$ en particular los limites de las zonas de invasibilidad estan relacionados con limitaciones energ\'eticas, dado que al momento de invadir la principal prioridad del invasor es alcanzar la cantidad de energ\'ia necesaria para subsistir en el ambiente, dicha limitaci\'on se observa en el hecho de que combinaciones con valores muy peque\~nos de $k_{\RC}$ y $k_{\CP}$ son excluidos en todos los casos explorados, y otra prueba de esto es el hecho que al aumentar la energ\'ia total disponible en el sistema o la tasa de adquisici\'on de energ\'ia, mediante aumentos en $k_0$ o $m_p$ , mayor proporci\'on de combinaciones en la comunidad $C_1$ son incluidas en las zonas de invasibilidad y coexistencia.\\


El par\'ametro $\phi$ va afecta el modo en el que los \emph{size ratios} $k_{ij}$ afectan a la tasa de consumci\'on por parte de los depredadores,como se observa en \ref{fig:efficiency} para valores de $k_{ij}$ mayores a 1 valores de $\phi$ elevados afectan negativamente la eficiencia de captura($\Pi$) y lo contrario se observa para $k_{ij} <1$. Para valores bajos(0.2 y 0.02 en nuestro caso) el efecto positivo que causa el que la presa sea mas grande(debido a un mayor movimiento (en caso de \emph{active} y \emph{sit and wait} strategies) y a un incremento en la energ\'ia disponible) domina al efecto negativo que causa $k_{ij}$ sobre la eficiencia de captura. Sin embargo para $\phi = 2$ tenemos que el efecto negativo es el que domina en valores elevados de $k_{ij}$. Esto se aprecia tanto por el hecho de que valores elevados de \emph{size ratios} son excluidos en el caso de $Z(I_{\C \to \R}),Z(I_{\PP \to \R}),Z(I_{\PP \to \C-\R})$ y adheridos en el caso de $Z(I_{\C \to \PP -\R})$ donde un valor elevado de $k_{\CP}$ implica una menor eficiencia en su captura por parte del deprador tope lo cual posibilita su invasi\'on al sistema. A su ves el valor de este par\'ametro controla la conexidad de las zonas de coexistencia, inserci\'on e invasibilidad mutua lo cual implica la posibilidad de tener una bimodalidad en los \emph{size ratios} encontrados en una comunidad compuesta por m\'as de un m\'odulo IGP asociadas incluso con una misma masa de depredador tope $m_P$.\\

Adem\'as observamos que cualitativamente las influencias de $m_p$ ,$k_0$ y $\phi$ sobre las zonas de invasibilidad y coexistencia son similares para las tres combinaciones de estrategia de forrajeo usadas, y como se observo en un an\'alisis previo\citep{pawar2012dimensionality} , la diferencia entre las zonas para las distintas combinaciones es menor comparada con la diferencia causada por distinta dimensi\'on de b\'usqueda. Esta \'ultima es causada por los cambios que se producen sobre la regi\'on de detecci\'on del depredador y la asociaci\'on de espacios de busqueda de dimensi\'on diferente con ambientes con diferentes tasas de productividad basal. \\


La zona de \emph{coexistencia inestable}(pero que la cual en principio puede formarse mediante una secuencia de invasiones), es extremadamente peque\~na lo cual estar\'ia indicando que para una parametrizaci\'on biol\'ogicamente realista si una combinaci\'on de masas presenta un equilibrio este con gan probalidad ser\'ia estable, es de resaltar que las comunidades asociadas con esta zona de coexistencia son las $C_3,C_4$ donde el depredador tope es mas peque\~no que sus presas cosa que concuerda con resultados previamente reportados(citar Martinez,Yodzis) que sugieren que depredadores peque\~nos al tener una mayor tasa de metabolismo por unidad de masa tienden a generar inestabilidades.El parametro $\phi$ tiene una influencia positiva sobre la estabilidad y al valor de 2 tenemos que toda la zona de coexistencia es estable, debido a como se dijo en el p\'arrafo anterior su influencia sobre la optimalidad de los size ratios, la cual restringe la inclusi\'on de valor extremos.


Es de resaltar que la zona de mutua invasibilidad es relativamente peque\~na lo que nos dice que existe una gran restricci\'on sobre las combinaciones de masas que har\'ian posible al modulo tener esta cualidad de ``\emph{bi-ensamblaje}'', nuestros resultados complementan resultados anteriores que relacionaron la distribuci\'on de masas con la estabilidad din\'amica del sistema(citar martinez) y a su ves muestran como un factor end\'ogeno puede modular la expresi\'on de los mecanimos de variaci\'on de la posici\'on tr\'ofica presente y que en cierta manera modular\'ia los efectos que producen factore ex\'ogenos que son los comunmente explorados(e.g tama\~no del ecosistema), y como se observa esta modulaci\'on es debido a cambios a nivel individual(metabolismo, movimiento) que escalan a propiedades interespec\'ificas(invasibilidad), esta forma de modular la longitud implica que para ambientes con caracter\'isticas f\'isicas similares pero con distinta distribuci\'on de masas en el pool regional (i.e pool de potenciales colonizadores) podemos esperar diferencias tanto en la historia de ensamblaje de la comunidad como en la m\'axima posici\'on tr\'ofica.

