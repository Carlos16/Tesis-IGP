\section{DISCUSION}

Nuestros resultados muestran que dependiendo de la combinanci\'on de masas presente en el m\'odulo IGP el curso del ensamblaje del m\'odulo puede tomar distintos caminos, en particular se observa que ciertas combinaciones restringen la \emph{adici\'on} de depredadores(tope o intermedio) o la \emph{inserci\'on} del depredador intermedio restringiendo de esta manera posibles cambios en la FCL observada en la comunidad. Estas observaciones complementan el trabajo realizado por \citet{holt2002food} en el cual sugiri\'o el hecho que procesos espaciales como \emph{colonizaci\'on} o \emph{extinci\'on} ser\'ian limitantes de FCL, para ser m\'as precisos estos procesos estar\'ian modulando la estructura de la \emph{red tr\'ofica} presente en la comunidad y a su vez la estructura en un instante determinado limita colonizaciones o extinciones futuras\citep{pawar2009community,holt2002food}.  Los resultados obtenidos en este trabajo sugieren que este \emph{feed-back} entre la estructura de la red y el procesos de ensamblaje estar\'ia a su vez influenciado por la distribuci\'on de masas presentes tanto en la comunidad receptora como en el pool de colonizadores.\\

Esta influencia se da principalmente por la forma como las relaciones entre masas afectan la energ\'ia total disponible y la tasa de p\'erdida de energ\'ia del invasor, la cual se refleja en la forma de los criterios encontrados.En los tres primeros casos tenemos que un para un par de $(k_\RC,k_\CP)$ la masa de las especies debe ser lo suficientemente grande para que se de la invasi\'on. Este comportamiento esta relacionado con el hecho que aumentos en la masa de las tres especies(las cuales aumentan de forma proporcional al aumentar $m_P$) implican aumentos en la capacidad de carga de $R$, tasas de ataque per ca\'pita( y en el caso de ambientes $3D$ aumentos en la tasas de ataque por unidad de masa) y disminuci\'on de tasas de p\'erdida de bimomasa.M\'as a\'un en el tercer caso tenemos que irrespectivamente de la masa de las especies interactuantes existen combinaciones de $(k_\RC,k_\CP)$ para las cuales es imposible que se de la invasi\'on, esto se debe a que para dichos size ratios tenemos que la tasa de consumici\'on de $P$ sobre $R$ y $C$ es insuficiente para cubrir los requerimientos energ\'eticos del invasor , lo que se pueda dar de dos formas: teniendo los valores al equilibrio  peque\~nos o que la eficiencia de captura del depredador sea muy baja(como se da para size ratios elevados y funciones unimodales). Es decir cuando hablamos de energ\'ia total disponible para el invasor no nos referimos a la energ\'ia total presente en el sistema sino a la fracci\'on que puede ser aprovechada por \'el.\\

En el cuarto caso(i.e la invasi\'on de $C$ a $P-R$,\emph{inserci\'on}), tenemos que existen size ratios en los cuales siempre es posible la invasi\'on de $C$, esto debido a que en estos casos o bien tenemos que el valor al equilibrio de $R$ es alto y $P$ bajo o la eficiencia de captura de $P$ sobre $C$ es baja. Fuera de estos size ratios tenemos un comportamiento cualitativo diferente dado que ahora el par de size ratios determinan un $m_P$ m\'aximo(tama\~no m\'aximo de masa para cada especie) sobre el cual la invasi\'on es imposible, esto se debe a un aumento relativo de la tasa de consumci\'on de $P$ sobre $C$ y la tasa de perdida de biomasa de $C$ con respecto a la ganancia de $C$ por la consumci\'on de $R$. Dado que para que se de la invasi\'on de $C$ sobre $P-R$ bajo nuestros supuestos necesitamos adem\'as que $P$ pueda invadir a $R$, tenemos que en el caso que estemos con size ratios para los cuales $\gamma_2 >0$ , la masa de cada especie esta atrapada entre una cota inferior que esta determinada por la invasi\'on de $P$ a $R$ ($\zeta_2$) y una cota superior determinada por $\zeta_4$, esta propiedad hace que dependiendo de las combinaciones de size ratios, el valor de $m_P$ sobre los cuales es posible la realizaci\'on de esta secuencia de invasiones es muy restringido.\\

Esta \'ultima propiedad hace que la zona de \emph{invasibilidad mutua} sea relativamente peque\~na, dado que ahora la cota inferior esta determinada por el m\'aximo de las cotas dadas por los tres primerios criterios. Esto nos dice que hay un mayor nivel de restricci\'on sobre las combinaciones de masas(y sus comunidades asociadas) con la propiedad de \emph{reemsamblarse} por m\'as de un camino de ensamblaje.\\

Otra manera en que las combinaciones de masas afectan a la $FCL$ es debido a su influencia sobre la regi\'on de coexistencia;dado que una condici\'on necesaria para tener una $MTP$ mayor a 2 es que las tres especies coexistan.\\

En este trabajo hemos dividido en base a $A$ la region de coexistencia(i.e equilibrio positivo) en $E_1$ y$E_2$ donde para $E_2$ tenemos que este equilibrio no puede formarse bajo una secuencia de ensamblaje con las suposiciones hechas y a su vez es inestable. $E_1$ comparte cualitativamente las caracter\'isticas con la zona de \emph{invasibilidad mutua}, existen size ratios sobre los cuales la coexistencia es imposible irrespectivamente de la masa de las especies presentes(asociados con el incumplimiento del criterio de invasi\'on de $P$ a $C-R$) y fuera de esos tenemos que para un par de size ratios fijo el $m_P$ para el cual se da la coexistencia esta atrapado(en el caso $c_\varepsilon$ sea suficientemente grande) entre dos cotas determinadas por los respectivos criterios de invasibilidad de $P$ y $C$. \\
Si bien la forma de las relaciones $\mu_i$ es independiente de la productividad basal, estrategia de forrajeo y dimensi\'on del espacio de b\'usqueda($h +1 - 2\beta >0$ tanto en 2D como 3D); est\'as afectan el valor de $\zeta_i$ y $\gamma_i$; y su respuesta frente a cambios en $(k_\RC,k_\CP)$.\\
En base a esto podemos discutir la influencia del nivel de productividad basal $\kappa_0$ , tanto sobre los criterios de invasibilidad como sobre la region de coexistencia $E_1$. Para todos los criterios aumentos en $\kappa_0$ disiminuyen el valor de la respectiva cota, esto se interpreta positivamente en los tres primeros casos , m\'as negativamente en el \'ultimo es decir, en ambientes con una productividad elevada la invasi\'on de $C$ a $P-R$ est\'a m\'as restringida. Esto repercute a su vez en la zona de coexistencia donde para una combinaci\'on de $m_P$ y size ratios dentro de $E_1$ existir\'a un valor de $\kappa_0$ por encima del cual lacoexistencia no ser\'ia posible, esto recobra resultados encontrados previamente\citep{holt1997theoretical}, sin embargo nos dice que el rango de producitividad basal sobre el cual puede coexistir el m\'odulo es dependiente de la combinaci\'on de masas presente, los size ratios determinan la cota superior y el $m_P$ la distancia a dicha cota $m_P$ cercanos a la cota inferior soportar\'an un mayor incremento en $\kappa_0$, sin embargo seran suceptibles a disminuciones en \'el. Teniendo por lo tanto que un una comunidad con $m_P$ situada en el punto medio entre ambas cotas ser\'ia la m\'as resistente a cambios(positivos o negativos) en la productividad basal.\\

De nuestros resultados tambi\'en podemos observar que hay una mayor influencia por parte de la dimensi\'on del espacio de b\'usqueda que por la distintas combinaci\'on de estrategias de forrajeo. Si bien en ambos casos hay cambios a nivel cuantitativo debido a su influencia sobre las tasas de ataque de los depredadores, la dimensi\'on del espacio de b\'usqueda tiende a afectar de forma \emph{cualitativa} ciertas relaciones(e.g comportamiendo de la condic\'on para $A >0$) entre los size ratios y las condiciones necesarias para la invasi\'on o coexistencia de las especies independientemente del valor de $\phi$, esto debido a que los cambios en los exponentes con los que escalan las tasas de ataque respecto a la masa y los size ratios tienen una mayor diferencia entre espacios de b\'usqueda de distinta dimensi\'on que entre distintas estrategias de forrajeo(si a su vez consideramos cambios en niveles de productividad la diferencia es a\'un mayor). \\

En conjunto las estrategias de forrajeo y la dimensi\'on del espacio de b\'usqueda influencian el valor de $\phi$ para el cual las funciones asociadas a los criterios(e.g tasa de consumci\'on al equilibrio) cambian de una forma mon\'otonta a una forma \emph{unimodal}, esto afecta la forma de adquisici\'on y p\'erdida de energ\'ia(excepto para el depredador tope) de las especies y por ende influencia las zonas de invasibilidad y coexistencia. En este sentido es de remarcar que para funciones \emph{unimodales} y un valor de $m_P$ fijo tenemos dos zonas diferenciadas de combinaciones de $k_\RC$ y $k_\CP$ dentro de la regi\'on de coexistencia, esto sugerir\'ia la existencia de dos \emph{tipos} de comunidades diferenciadas principalmente por el valor de $k_\CP$(mayor o menor que 1). \\

Dada la influencia que tiene la masa corporal de las especies sobre la tasa de metabolismo, y su subsecuente influencia sobre para\'ametros tanto \emph{intra} e \emph{inter-poblacionales}\citep{savage2004predominance,brown2004toward,west1997general,savage2004effects,pawar2012dimensionality,mcgill2006allometric,peters1986ecological,kiltie2000scaling,yodzis1992body} es plausible pensar que la distribuci\'on de masas tendr\'ia un efecto sobre los mecanismos de invasibilidad y las zonas de coexistencia del m\'odulo IGP , irrespectivamente de las particularidades del modelo usado(e.g respuesta funcional Tipo I).\\



