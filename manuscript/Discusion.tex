\section{DISCUSION}

Nuestros resultados muestran que dependiendo de la combinanci\'on de masas presente en el m\'odulo IGP el curso del ensamblaje del m\'odulo puede tomar distintos caminos, en particular se observa que ciertas combinaciones restringen la \emph{adici\'on} de depredadores(tope o intermedio) o la \emph{inserci\'on} del depredador intermedio restringiendo de esta manera posibles cambios en la LCT observada en la comunidad\citep{TP2007proximate}. Estas observaciones complementan el trabajo realizado por \citet{holt2002food} en el cual sugiri\'o el hecho que procesos espaciales como \emph{colonizaci\'on} o \emph{extinci\'on} ser\'ian limitantes de LCT, para ser m\'as precisos estos procesos estar\'ian modulando la estructura de la \emph{red tr\'ofica} presente en la comunidad, y a su vez la estructura en un instante determinado limita colonizaciones o extinciones futuras \citep{pawar2009community,holt2002food}. Los resultados obtenidos en este trabajo sugieren que este \emph{feed-back} entre la estructura de la red y el proceso de ensamblaje estar\'ia a su vez influenciado por la distribuci\'on de masas presentes tanto en la comunidad receptora como en el pool de colonizadores.\\

Esta influencia se da principalmente por la forma como las relaciones entre masas afectan la energ\'ia total disponible al instante de la invasi\'on y la tasa de p\'erdida de energ\'ia del invasor \citep{pawar2012dimensionality}, la cual se refleja en la forma de los criterios encontrados. A su vez es de resaltar la relativamente baja medida de la zona de invasibilidad mutua, lo cual sugiere una relaci\'on entre la masa de las especies y la historia de ensamblaje en la determinaci\'on del estado final del m\'odulo \citep{fukami2004}.\\

Las relaciones entre masas tambi\'en afectan a la LCT debido a su influencia sobre la zona de coexistencia(i.e., equilibrio positivo); dado que una condici\'on necesaria para tener una $MTP$ mayor a 2 es que las tres especies coexistan. Dentro de la zona de coexistencia, desviaciones del valor lineal de $3$ para la MTP estan influenciados por una relaci\'on puramente entre $k_\RC$ y $k_\CP$ y dentro de gran parte de la zona de coexistencia se observa que los valores ser\'ian intermedios. Esto resalta la influencia del para de raz\'on de masas sobre el mecanismo de omnivorismo \citet{TP2007proximate}. \\

Para el caso donde es posible encontrar caminos de ensamblaje (denominada $E_1$ y determinada para $A > 0$) hacia el estado de coexistencia, tenemos que existen combinaciones de $k_\RC, k_\CP$ para las cuales sin importar la masa de las especies no es posible la coexistencia, esto se da debido a la insuficiente energ\'ia disponible para el depredador tope independientemente del valor de productividad basal del ambiente (citar productivity hypothesis). Fuera de estos casos, combinaciones de masas que satisfagan los criterios nos garantizan adem\'as que el equilibrio sea localmente estable; teniendo en cuenta que esta relaci\'on no existe para valores arbitrarios de par\'ametros \citep{holt1997theoretical}, determinar el valor de los par\'ametros en base a la relaci\'on de la masa de las especies sobre la tasa de metabolismo y movimiento aumenta la ``estabilidad'' del m\'odulo \citep{brose2006allometric,tangpawarallesina2014}. \\

Estas relaciones nos sugieren que cambios en la masa de las especies interactuantes, por ejemplo por factores evolutivos, afectar\'ian las propiedades tanto de ensamblaje como la feasibilidad del equilibrio pudiendo en ciertos casos (dependiendo de la magnitud) provocar el cambio de estado en el m\'odulo(e.g., pasar de un estado de coexistencia estable a otro donde la coexistencia en equilibrio no es posible) con subsequente(s) extinciones.\\

La forma de las relaciones $\mu_i$ depende del supuesto $ h + 1 - 2\beta > 0$, que se cumple con el valor \emph{t\'ipico} reportado para estos par\'ametros \citep{pawar2012dimensionality,brown2004toward}. Esta forma es independiente de la productividad basal, estrategia de forrajeo y dimensi\'on del espacio de b\'usqueda ($h +1 - 2\beta > 0$ tanto en 2D como 3D); sin embargo \'estas afectan el valor de $\zeta_i$ y $\gamma_i$; y su respuesta frente a cambios en $(k_\RC,k_\CP)$.\\

En particular podemos discutir la influencia del nivel de productividad basal $\kappa_0$, tanto sobre los criterios de invasibilidad como sobre la region de coexistencia $E_1$. Para todos los criterios aumentos en $\kappa_0$ disiminuyen el valor de la respectiva cota, esto se interpreta positivamente en los tres primeros casos, mas negativamente en el \'ultimo es decir, en ambientes con una productividad elevada la invasi\'on de $C$ a $P-R$ est\'a m\'as restringida. Esto repercute a su vez en la zona de coexistencia donde para una combinaci\'on de $m_P$ y raz\'on de masas dentro de $E_1$ existir\'a un valor de $\kappa_0$ por encima del cual la coexistencia no ser\'ia posible, esto recobra resultados encontrados previamente\citep{holt1997theoretical}, sin embargo nos dice que el rango de producitividad basal sobre el cual puede coexistir el m\'odulo es dependiente de la combinaci\'on de masas presente, los raz\'on de masas determina la cota superior mientras que $m_P$ determina la distancia a dicha cota. $m_P$ cercanos a la cota inferior soportar\'an un mayor incremento en $\kappa_0$, sin embargo ser\'an suceptibles a disminuciones en $\kappa_0$. Teniendo por lo tanto que un una comunidad con $m_P$ situada en el punto medio entre ambas cotas ser\'ia la m\'as resistente a cambios(positivos o negativos) en la productividad basal $\kappa_0$. Esto nos dice que estudios que deseen evaluar la influencia de la productividad basal sobre la din\'amica de este tipo de comunidades \citep{diehl2001intraguild,takimoto2007intraguild,TP2007proximate} deben tomar en cuenta en el dise\~no la masa de las especies con las que van a trabajar.\\

En nuestros resultados tambi\'en observamos que hay una mayor influencia por parte de la dimensi\'on del espacio de b\'usqueda que por la distintas combinaciones de estrategias de forrajeo \citep{pawar2012dimensionality}. Si bien en ambos casos hay cambios a nivel cuantitativo debido a su influencia sobre las tasas de ataque de los depredadores, la dimensi\'on del espacio de b\'usqueda tiende a afectar de forma \emph{cualitativa} ciertas relaciones (e.g comportamiendo de la condic\'on para $A > 0$) entre las razones de masas y las condiciones necesarias para la invasi\'on o coexistencia de las especies independientemente del valor de $\phi$, esto debido a que los cambios en los exponentes con los que escalan las tasas de ataque respecto a la masa y los raz\'on de masas tienen una mayor diferencia entre espacios de b\'usqueda de distinta dimensi\'on que entre distintas estrategias de forrajeo (si a su vez consideramos cambios en niveles de productividad la diferencia es a\'un mayor). \\

En conjunto las estrategias de forrajeo y la dimensi\'on del espacio de b\'usqueda influencian el valor de $\phi$ para el cual las funciones asociadas a los criterios(e.g tasa de consumo al equilibrio) cambian de una forma mon\'otonta a una forma \emph{unimodal}, esto afecta la forma de adquisici\'on y p\'erdida de energ\'ia (excepto para el depredador tope) de las especies y por ende influencia las zonas de invasibilidad y coexistencia. En este sentido es de remarcar que para funciones \emph{unimodales} y un valor de $m_P$ fijo tenemos dos zonas diferenciadas de combinaciones de $k_\RC$ y $k_\CP$ dentro de la regi\'on de coexistencia, esto sugiere la existencia de dos \emph{tipos} de comunidades diferenciadas principalmente por el valor de $k_\CP$ (mayor o menor que 1). \\

Los datos emp\'iricos muestran que el modelo no explica con precisi\'on la coexistencia de especies para valores de productividad altos. Como se detalla en resultados las explicaciones a esto son diversas, siendo las m\'as atrayente la extensi\'on a metacomunidades\citep{Takimoto2012} y la incorporaci\'on de presas adicionales para los consumidores del m\'odulo \citep{holt2007alternative}. Es de recordar que hemos trabajado con datos de dimensi\'on de b\'usqueda $3D$ y se ha reportado previamente que ambientes de b\'usqueda $3D$ ser\'ian m\'as inestables \citep{pawar2012dimensionality} y por lo tanto esto imposibilitaria la existencia de equilibrio positivo en estos casos. Pese a esto la relaci\'on existente de los diversos criterios con la productividad basal es la esperada \citet{holt1997theoretical,diehl2000effects}.

Dada la influencia de la masa corporal de las especies sobre la tasa de metabolismo, y su subsecuente influencia sobre para\'metros tanto \emph{intra} e \emph{inter-poblacionales} \citep{savage2004predominance,brown2004toward,west1997general,savage2004effects,pawar2012dimensionality,mcgill2006allometric,peters1986ecological,kiltie2000scaling,yodzis1992body} es plausible pensar que la distribuci\'on de masas tendr\'ia un efecto sobre los mecanismos de invasibilidad y las zonas de coexistencia del m\'odulo IGP y por ende sobre la longitud de la cadena tr\'ofica, sin importar las particularidades del modelo usado (e.g respuesta funcional Tipo I). A su vez se observa que este factor intr\'inseco condiciona el accionar de factores ambientales sobre las propiedades del m\'odulo(e.g influencia de la productividad basal sobre la zona de coexistencia) y por ende es un factor a tomar en cuenta en la discusi\'on sobre los determinantes de propiedades estructurales del sistema como la longitud de la cadena tr\'ofica \citep{takimoto2013environmental,post2002long}.





