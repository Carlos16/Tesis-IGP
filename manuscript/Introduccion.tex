\subsection{INTRODUCCION}

Desde el descubrimiento y cuantificaci\'on de la relaci\'on existente entre la masa corporal de una especie y su tasa metab\'olica,lo que se denomin\'o \emph{kleiber's law} \citep{kleiber1961fire}, y la posterior sint\'esis de los registros emp\'iricos realizada porPeters \citep{peters1986ecological}; se suger\'ia una relaci\'on importante entre la masa corporal y la ecolog\'ia de las especies.Posteriormente Yodzis e Innes en su \emph{seminal paper, Body size and consumer resource dynamics}\citep{yodzis1992body} introdujeron estas ideas en los modelos depredador-presa,afectando particularmente la parametrizaci\'on de dichos modelos.Sin embargo formalizaci\'on de estas ideas bajo una teor\'ia se retras\'o hasta principios de este siglo bajo la denominada , \emph{Toer\'ia metab\'olica de la ecolog\'ia}(MTE), desarrollada entre otro por James Brown y Geoffrey West\citep{brown2004toward},a partir de ella se empez\'o a percibir de forma m\'as general la importancia del metabolismo, y por ende de la masa corporal, sobre distintos procesos ecol\'ogicos. Dada la caracter\'istica universal que tiene la relaci\'on entre la masa corporal y el metabolismo, la MTE y sus futuras extensiones tienen el potencial de convertirse en una teor\'ia que unifique las distintas \'areas de la ecolog\'ia.\\
En los \'ultimos a\~nos se han extendido los efectos de la masa a procesos comunitarios, en particular en el estudio de redes tr\'oficas, donde se ha sugerido particularmente que la masa de las especies interactuantes afectan la estabilidad, persistencia y funcionamiento de la red, debido a su influencia sobre la fuerza de interacci\'on entre especies y la velocidad de flujo de energ\'ia y masa a traves de ella. \citep{varios, citar luego}

Sin embargo a\'un no ha habido mucho \'enfasis en de que manera la masa corporal afecta el ensamblaje o estructuraci\'on de las comunidades ecolo\'ogicas, en particular de las \emph{redes tr\'oficas}, la investigaci\'on en dicha \'area es fundamental si es que se desea lograr un completo entendimiento de las comunidades ecol\'ogicas ,y desde desde el punto de vista aplicado para evaluar la resiliencia y robustes de las comunidades, y fomentar programas de recolonizaci\'on en lugares altamente perturbados.\\

En este trabajo nos enfocamos en dicha direcci\'on y para ello analizamos el m\'odulo de tres especies denomidado \emph{Intraguild predation},el cual fue descrito en \cite{polis1989ecology} y el primer an\'alisis te\'orico desarrollado por \cite{holt1997theoretical}.Derivamos las relaciones que existen entre la masa corporal de las especies que constituyen el m\'odulo y las condiciones necesarias para la expresi\'on de los mecanimos de ensamblaje del m\'odulo. \\
Adem\'as, dada la conexi\'on existente entre el ensamblaje del m\'odulo y lo descrito por \cite{TP2007proximate} como \emph{mecanimos proximales de variaci\'on estructural de las cadenas tr\'oficas}, derivamos los efectos que la masa de las especies, y m\'as a\'un la raz\'on de masas presa-depredador, tienen sobre la longitud de las cadenas tr\'oficas. \\

La longitud de las cadenas tr\'oficas es una propiedad estructural importante de los ecosistemas ,que afecta distintas funciones ecol\'ogicas de \'este (e.g. reciclaje de nutrienes, patr\'on de efectos cascada), por lo cual  la observaci\'on de la variaci\'on de su longitud despert\'o y sigue despertando el inter\'es de muchos investigadores\citep{ulanowicz2014limits,borrelli2014there}, los cuales han tratado de determinar los factores que determinan esta longitud; si bien se han propuesto factores ambientales como posibles determinantes de la longitud: R\'egimen de disturbancias, Tama\~no del ecosistema y Nivel de productividad \citep{post2002long,takimoto2013environmental}; a\'un no se ha llegado a una conclusi\'on definitiva sobre los mecanismos que regulan esta longitud \citep{sterner1997enigma,takimoto2013environmental}. En los \'ultimos a\~nos se a pasado de determinar qu\'e factor es m\'as importante a determinar c\'omo los distintos factores interactuan entre s\'i y bajo que contexto son m\'as o menos importantes \citep{post2002long}. En nuestro caso observamos que muy aparte de los factores externos , factores internos a las comunidades tambi\'en influencian esta longitud.

