\section{INTRODUCCION}

La longitud de las cadenas tr\'oficas(en adelante LCT) es una propiedad estructural importante de los ecosistemas ,que afecta distintas funciones ecol\'ogicas de \'este (e.g., reciclaje de nutrienes \citep{deangelis92}, patr\'on de efectos cascada \citep{terborgh2013trophic}).\\
A pesar de la gran variabilidad existente en el n\'umero de especies en distintas comunidades, las longitudes encontradas oscilan principalmente en el rango de 2 a 4 pasos\citep{elton1927animal,pimm1978feeding,yodzis1981structure,williams2004limits}, esta relativa limitaci\'on en su variaci\'on implica que existen factores que restringen las longitudes posibles.\\

Si bien se han propuesto diversos factores como limitantes: r\'egimen de disturbancias, tama\~no del ecosistema y nivel de productividad \citep{post2002long,takimoto2013environmental}; a\'un no se ha llegado a una conclusi\'on definitiva sobre los mecanismos que regulan esta longitud \citep{sterner1997enigma,takimoto2013environmental} y en los \'ultimos a\~nos se a pasado de determinar qu\'e factor es m\'as importante a determinar c\'omo los distintos factores interactuan entre s\'i y bajo que contexto son m\'as o menos importantes \citep{post2002long}.\\

En nuestro caso exploramos la influencia de la distribuci\'on de masas presente en la comunidad sobre la expresi\'on de los mecanismos de variaci\'on de las cadenas tr\'oficas \ (\emph{mecanimos proximales de variaci\'on estructural de las cadenas tr\'oficas sensu \cite{TP2007proximate}}). Para ello analizamos el m\'odulo de tres especies denomidado \emph{predaci\'on intra-gremial}, el cual fue descrito en \cite{polis1989ecology} y el primer an\'alisis te\'orico desarrollado por \cite{holt1997theoretical}. Debido a la conexi\'on existente entre los mecanimos de variaci\'on de la LCT y los mecanismos de ensamblaje del m\'odulo\citep{TP2007proximate} este an\'alisis es equivalente a estudiar las relaciones existentes entre la masa corporal de las especies que constituyen el m\'odulo y, las condiciones para la expresi\'on de los mecanismos de ensamblaje del m\'odulo y la zona de coexistencia de las tres especies. Una relaci\'on entre procesos relacionados con el ensamblaje y la LCT fue reportada previamente por \citet{holt2002food}.\\

La influencia de la distribuci\'on de masas ser\'a explorada bajo la premisa de la relaci\'on existente entre la masa corporal de las especies y su tasa de metabolismo. La relaci\'on de la masa corporal con la ecolog\'ia de las especies data desde el descubrimiento y cuantificaci\'on de la relaci\'on existente entre la masa corporal de una especie y su tasa metab\'olica, lo que se denomin\'o \emph{ley de Kleiber} \citep{kleiber1961fire}, y la posterior s\'intesis de registros emp\'iricos de dicha relaci\'on en distintas especies realizada por Peters \citep{peters1986ecological}. Un enfoque m\'as moderno es el dado por la denominada \emph{Teor\'ia metab\'olica de la ecolog\'ia}(MTE), desarrollada entre otros por James Brown y Geoffrey West\citep{brown2004toward}, donde se da un fundamento te\'orico para esta relaci\'on\citep{west1997general, savage2004predominance} y se formaliza la relaci\'on entre el metabolismo y procesos ecol\'ogicos a distinto nivel\citep{savage2004effects,brown_metabolic_book}.\\

En los \'ultimos a\~nos se han explorado los efectos de la masa corporal sobre distintos procesos comunitarios, en particular en el estudio de redes tr\'oficas, donde se ha demostrado que la masa de las especies interactuantes afectan la estabilidad, persistencia y funcionamiento de la red, debido a su influencia sobre la fuerza de interacci\'on entre especies y la velocidad de flujo de energ\'ia y masa a traves de ella. \citep{brose2006allometric,mccann2011food}\\

Pese a ello a\'un no ha habido mucho \'enfasis en la relaci\'on de la masa corporal con el ensamblaje o estructuraci\'on de las comunidades ecol\'ogicas, en particular de las \emph{redes tr\'oficas}, la investigaci\'on en dicha \'area es fundamental si es que se desea lograr un completo entendimiento de las comunidades ecol\'ogicas, y desde desde el punto de vista aplicado para evaluar la resiliencia y robustes de las comunidades, y fomentar programas de recolonizaci\'on en lugares altamente perturbados.\\

En este trabajo, derivamos criterios de invasibilidad y coexistencia en base a las masas de las especies presente en el m\'odulo de depredaci\'on intragremial y exploramos de esta manera su relaci\'on con los mecanismos de ensamblaje del m\'odulo y subsequentemente con la LCT observada en \'el.

