\section{INTRODUCCION}

La longitud de las cadenas tr\'oficas(en adelante FCL) es una propiedad estructural importante de los ecosistemas ,que afecta distintas funciones ecol\'ogicas de \'este (e.g., reciclaje de nutrienes, patr\'on de efectos cascada).\\
A pesar de la gran variabilidad existente en el n\'umero de especies en distintas comunidades, las longitudes encontradas oscilan principalmente en el rango de 2 a 4 pasos\citep{elton1927animal,pimm1978feeding,yodzis1981structure,williams2004limits}, esta relativa limitaci\'on en su variaci\'on implica que existen factores que restringen las longitudes posibles.\\
Si bien se han propuesto diversos factores como limitantes: r\'egimen de disturbancias, tama\~no del ecosistema y nivel de productividad \citep{post2002long,takimoto2013environmental}; a\'un no se ha llegado a una conclusi\'on definitiva sobre los mecanismos que regulan esta longitud \citep{sterner1997enigma,takimoto2013environmental} y en los \'ultimos a\~nos se a pasado de determinar qu\'e factor es m\'as importante a determinar c\'omo los distintos factores interactuan entre s\'i y bajo que contexto son m\'as o menos importantes \citep{post2002long}.\\

En nuestro caso exploramos la influencia de la distribuci\'on de masas presente en la comunidad sobre la expresi\'on de los mecanismos de variaci\'on de las cadenas tr\'oficas \ ( \emph{mecanimos proximales de variaci\'on estructural de las cadenas tr\'oficas sensu \cite{TP2007proximate}}). Para ello analizamos el m\'odulo de tres especies denomidado \emph{predaci\'on intra-clan},el cual fue descrito en \cite{polis1989ecology} y el primer an\'alisis te\'orico desarrollado por \cite{holt1997theoretical}. Debido a la conexi\'on existente entre los mecanimos de variaci\'on de la FCL y los mecanismos de ensamblaje del m\'odulo\citep{TP2007proximate} este an\'alisis es equivalente a estudiar las relaciones existentes entre la masa corporal de las especies que constituyen el m\'odulo y, las condiciones para la expresi\'on de los mecanismos de ensamblaje del m\'odulo y la zona de coexistencia de las tres especies. Una relaci\'on entre procesos relacionados con el ensamblaje y la FCL ha sido reportado previamente por \citet{holt2002food}.\\

La influencia de las masas ser\'a explorada bajo la premisa de la relaci\'on existente entre la masa de las especies y su tasa de metabolismo. Desde el descubrimiento y cuantificaci\'on de la relaci\'on existente entre la masa corporal de una especie y su tasa metab\'olica,lo que se denomin\'o \emph{kleiber's law} \citep{kleiber1961fire}, y la posterior sint\'esis de los registros emp\'iricos realizada por Peters \citep{peters1986ecological}; se suger\'ia una relaci\'on importante entre la masa corporal y la ecolog\'ia de las especies. Posteriormente Yodzis e Innes en su \emph{seminal paper, Body size and consumer resource dynamics}\citep{yodzis1992body} introdujeron estas ideas en los modelos depredador-presa,afectando particularmente la parametrizaci\'on de dichos modelos. Sin embargo formalizaci\'on de estas ideas bajo una teor\'ia se retras\'o hasta principios de este siglo bajo la denominada , \emph{Teor\'ia metab\'olica de la ecolog\'ia}(MTE), desarrollada entre otros por James Brown y Geoffrey West\citep{brown2004toward}, a partir de ella se empez\'o a percibir de forma m\'as general la importancia del metabolismo, y por ende de la masa corporal, sobre distintos procesos ecol\'ogicos. Dada la caracter\'istica universal que tiene la relaci\'on entre la masa corporal y el metabolismo, la MTE y sus futuras extensiones tienen el potencial de convertirse en una teor\'ia que unifique las distintas \'areas de la ecolog\'ia.\\

En los \'ultimos a\~nos se han extendido los efectos de la masa a procesos comunitarios, en particular en el estudio de redes tr\'oficas, donde se ha sugerido particularmente que la masa de las especies interactuantes afectan la estabilidad, persistencia y funcionamiento de la red, debido a su influencia sobre la fuerza de interacci\'on entre especies y la velocidad de flujo de energ\'ia y masa a traves de ella. \citep{brose2006allometric,mccann2011food}

Sin embargo a\'un no ha habido mucho \'enfasis en de que manera la masa corporal afecta el ensamblaje o estructuraci\'on de las comunidades ecolo\'ogicas, en particular de las \emph{redes tr\'oficas}, la investigaci\'on en dicha \'area es fundamental si es que se desea lograr un completo entendimiento de las comunidades ecol\'ogicas ,y desde desde el punto de vista aplicado para evaluar la resiliencia y robustes de las comunidades, y fomentar programas de recolonizaci\'on en lugares altamente perturbados.\\

Con este trabajo pretendemos dar un paso en dicha direcci\'on , exponiendo la influencia de la masa corporal-lo cual no ha sido considerado previamente- sobre la variaci\'on de la FCL.

