\section{Conclusiones}

La distribuci\'on de masas presente en una comunidad modula la longitud de la cadena tr\'ofica, debido a su influencia sobre los mecanismos de ensamblaje y coexistencia de las especies, las relaciones existentes se basan en la forma en que las masas afectan la tasa de p\'erdida y adquisici\'on de energ\'ia de las especies involucradas.\\
La forma de las relaciones ($\mu_i$) es independiente de la productividad del ambiente, la dimensi\'on del espacio de b\'usqueda y la combinaci\'on de estrategias de forrajeo; sin embargo estas afectan el valor de $\zeta_i$ y $\gamma_i$. La acci\'on de los dos \'ultimos a su vez puede cambiar la forma de la respuesta de $\zeta_i$ y $\gamma_i$ respecto a cambios en $(k_\RC,k_\CP)$ y se observa un mayor efecto por parte de la dimensi\'on del espacio de b\'usqueda.\\
La influencia de la distribuci\'on de masas sobre el curso del ensamblaje de la comunidad (y la red tr\'ofica asociada) presenta un \'area de investigaci\'on muy prometedora y que a\'un no recibe la adecuada atenci\'on, esto a su vez podr\'ia acoplarse a la influencia que dicha distribuci\'on presenta sobre el funcionamiento del ecosistema a lo largo del proceso de ensamblaje. Resultados asociados a estos estudios ser\'ian de gran importancia no s\'olo desde el punto de vista te\'orico sino aplicado.





