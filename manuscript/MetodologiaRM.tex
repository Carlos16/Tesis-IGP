\section{METODOLOGIA}

\subsection{Modelaci\'on matem\'atica}



Tomando como base el m\'odulo de tres especies  IGP - \emph{Intraguild predation}-(descrito en \citealt{polis1989ecology,polis1992intraguild}),usamos un modelo matem\'atico\citep{holt1997theoretical} para describir las interacciones entre las especies componentes(Recurso, consumidor intermedio(IG prey) y depredador tope(IG Predator) ). Este m\'odulo ha sido sugerido como un sistema de complejidad suficiente para incorporar distintos mecanimos que provocan variaci\'on en la longitud de las cadenas tr\'oficas : Inserci\'on, Adici\'on y Omnivorismo \citep{TP2007proximate}.
\subsubsection{Forma general}
La din\'amica del systema est\'a gobernada ,de forma general,por el siguiente sistema de ecuaciones diferenciales: \\
Sea $ N= (N_1,N_2,N_3) = (\dot{R} , \dot{C} , \dot{P})  $
\begin{equation}\label{eq:Gsystem}
\begin{aligned}
&\dot{R} = F(R) -G(R,C)-H_{\R}(R,C,P)  \\
&\dot{C} = \epsilon_1 G(R,C)-H_{\C}(R,C,P) - q_2 C  \\
&\dot{P} = \epsilon_2 H_{\R}(R,C,P) +\epsilon_3 H_{\C}(R,C,P) -q_1 P
\end{aligned}
\end{equation}
\subsubsection*{Donde:}
\begin{tabular}{l@{:}p{5.8in}}
$R$\ &   Densidad de biomasa del recurso $R$.\\
$C$\ &  Densidad de biomasa del consumidor intermedio(IG prey) $C$.\\
$P$\ &  Densidad de biomasa del depredador tope(IG predator) $P$.\\
$F$\ &  Funci\'on que describe la din\'amica poblacional del recurso $R$ en ausencia de depredadores.\\
$G$\ &  Funci\'on que describe la depredaci\'on ejercida por el consumidor $C$ sobre el recurso $R$.\\
$H_{\R}$\ &  Funci\'on que describe la depredaci\'on ejercida por el depredador tope $P$ sobre el recurso $R$.\\
$H_{\C}$\ &  Funci\'on que describe la depredaci\'on ejercida por el depredador tope $P$ sobre el consumidor $C$.\\
$\epsilon_1$\ &  Eficiencia de conversi\'on de biomasa del recurso $R$ en biomasa del consumidor intermedio $C$.\\
$\epsilon_2$\ &  Eficiencia de conversi\'on de biomasa del recurso $R$ en biomasa del depredador tope $P$.\\
$\epsilon_3$\ &  Eficiencia de conversi\'on de biomasa del consumidor $C$ en biomasa del depredador tope $P$.\\
$q_1$\ &  Tasa de p\'erdida de biomasa por unidad de masa del consumidor intermedio $C$.\\
$q_2$\ &  Tasa de p\'erdida de biomasa por unidad de masa del depredador tope $P$.\\
\end{tabular}
\subsubsection{Forma espec\'ifica I - \emph{Lotka-Volterra}}
En este caso asumimos que el recurso tiene un crecimiento log\'istico y que la tasa de consumci\'on per c\'apita de biomasa escala linearmente con la densidad de biomasa del recurso, i.e. usamos una respuesta funcional Tipo I \citep{gotelliprimer}.\\
Esta forma es una modificaci\'on del modelo Lotka-Volterra \citep{gotelliprimer}.\\
\mbox{}\\
Definimos:
\begin{equation}\label{eq:p1}
\begin{aligned}
&r := r(m_r,T_r,\vartheta_r)&\\ 
&K := K(m_r,T_r,\vartheta_r)&\\
&\alpha_{1} := \alpha_{1}(m_\R,m_c,T_r,T_c,D_r,f_\C,\vartheta_r,\vartheta_c)&\\
&\alpha_{2} := \alpha_{2}(m_\R,m_\PP,T_r,T_p,D_r,f_{\PP_2},\vartheta_r,\vartheta_p)&\\
&\alpha_{3} := \alpha_{3}(m_\C,m_\PP,T_c,T_p,D_c,f_{\PP_3},\vartheta_c,\vartheta_p)& \\
&q_1 := q_1(m_c,T_c,\vartheta_c)& \\ 
&q_2 := q_2(m_p,T_p,\vartheta_p)&\\
&F(R):= rR(1-R/K)& \\
&G(R,C):= (\alpha_1/m_c)RC &\\ 
&H_{\R}(R,C,P):= (\alpha_2/m_p)PR&\\
&H_{\C}(R,C,P):= (\alpha_2/m_p)PC&
\end{aligned}
\end{equation}

Reemplazando \eqref{eq:p1} en \eqref{eq:Gsystem} tenemos:

\
\begin{equation}
\begin{aligned} 
\dot{R} &= R\left[r(1-R/K)- (\alpha_{1}/m_\C) C -(\alpha_{2}/m_\PP) P \right] \\
\dot{C} &= C \left[\epsilon_1 (\alpha_{1}/m_\C) R - (\alpha_{3} /m_\PP) P - q_1\right] \\
\dot{P} &= P \left[\frac{1}{m_\PP}\left(\epsilon_2 \alpha_{2}R + \epsilon_3 \alpha_{3}C \right) - q_2 \right]
\end{aligned}
\end{equation}

\subsubsection*{Donde:}
\begin{longtable}{l@{:}p{5.8in}}
$r$  \ & Tasa intr\'inseca de producci\'on de biomasa del Recurso $R$.\\
$K$  \  &Capacidad de carga (en biomasa) del recurso $R$.\\
$\alpha_1$  \ & Tasa per c\'apita de b\'usqueda y captura de biomasa del depredador intermedio $C$ sobre  el recurso basal $R$.\\
$\alpha_2$ \ & Tasa per c\'apita de b\'usqueda y captura de biomasa del depredador tope $P$ sobre  el recurso basal $R$.\\
$\alpha_3$ \  &Tasa per c\'apita de b\'usqueda y captura de biomasa del depredador tope $P$ sobre  el depredador intermedio $C$.\\
$m_\R$  \ & Masa de un individuo ``t\'ipico'' del recurso $R$.\\
$m_\C$  \ & Masa de un individuo ``t\'ipico'' del consumidor intermedio $C$.\\
$m_\textit{\tiny P}$  \ & masa de un individuo ``t\'ipico'' del depredador tope $P$.\\
$T_J$ \ & Temperatura corporal promedio de la especie $J$. $J = P,C,R$ \\
$\vartheta_J$ \ & Clase metab\'olica de la especie $J$ . $J = P,C,R$ \\
$f_\C$ \ & Estrategia de forrajeo de la consumidor intermedio $C$ .\\
$f_{\PP_2}$ \ & Estrategia de forrajeo del depredador tope $P$ sobre $R$ .\\
$f_{\PP_3}$ \ & Estrategia de forrajeo del depredador tope $P$ sobre $C$ .\\

\end{longtable}

Los par\'ametros y funciones no mencionados mantienen la descripci\'on dada en el caso general.
\subsubsection{Forma espec\'ifica II - \emph{Rozenweigh-MacArthur}}
Una ligera modificaci\'on del modelo anterior, donde consideramos que los depredadores tienen un punto de saciedad, es decir la tasa de consumci\'on de biomasa por unidad de depredador tiene un punto m\'aximo \emph{finito}, para modelar esto usamos una respuesta funcional tipo II \citep{gotelliprimer}, la cual incorpora un par\'ametro para describir el tiempo que demoran los depredadores en perseguir,subyugar e ingerir a la presa una vez que esta haya sido detectada y donde la captura fue existosa\citep{brose2010body}.\\
Definimos:
\begin{equation}\label{eq:p2}
\begin{aligned}
&\hat{\alpha}_1 = \alpha_1/m_\C&\\
&\hat{\alpha}_2 = \alpha_2/m_\PP&\\
&\hat{\alpha}_3 = \alpha_3/m_\PP&\\
&t_{h_\C} := t_{h_{\C}} ( m_\C , T_c, \vartheta_c)\\
&t_{h_\PP} := t_{h_{\PP}} ( m_\PP , T_p, \vartheta_p)\\
&G(R,C):= \frac{\hat{\alpha_1}RC}{1+t_{h_\C}\alpha_1 R} &\\ 
&H_{\R}(R,C,P):= \frac{\hat{\alpha_1}RP}{1+t_{h_\PP}(\alpha_2 R+\alpha_3C)}\\
&H_{\C}(R,C,P):= \frac{\hat{\alpha_1}CP}{1+t_{h_\PP}(\alpha_2 R+\alpha_3C)}
\end{aligned}
\end{equation}

Reemplazando \eqref{eq:p2} en \eqref{eq:Gsystem} :
\begin{equation}
\begin{aligned} 
\dot{R} &= R\left[r(1-R/K)- \frac{\hat{\alpha}_1 C}{1+t_{h_\C} \alpha_1 R} -\frac{\hat{\alpha}_2 P}{1+t_{h_\PP} (\alpha_3C+ \alpha_2R)} \right] \\
\dot{C} &= C \left[\frac{\epsilon_1 \hat{\alpha}_1 R}{1+t_{h_\C}\alpha_1 R} - \frac{\hat{\alpha}_3 P}{1+t_{h_\PP} (\alpha_3C+ \alpha_2R)} - q_1\right] \\
\dot{P} &= P \left[\frac{1}{1+t_{h_\PP} (\alpha_3C+ \alpha_2R)} (\epsilon_2 \hat{\alpha}_2R +\epsilon_3 \hat{\alpha}_3C) - q_2 \right]
\end{aligned}
\end{equation}

\subsubsection*{Donde:}
\begin{tabular}{l@{:}l}
$t_{h_\C}$ & \  Tiempo de manejo por unidad de masa de recurso, del consumidor intermedio $C$.\\
$t_{h_\C}$ & \  Tiempo de manejo por  unidad de masa de recurso, del depredador tope $P$.\\
\end{tabular}

\subsubsection{Parametrizaci\'on}
Usamos relaciones alom\'etricas derivadas previamente en la literatura basadas en relaciones biomec\'anicas y bioenerg\'eticas \citep{savage2004predominance,brown2004toward,west1997general,savage2004effects,pawar2012dimensionality,mcgill2006allometric,peters1986ecological,kiltie2000scaling,yodzis1992body} para exponer expl\'icitamente la variaci\'on de los par\'ametros de los modelos usados con respecto a la masa corporal de las especies interactuantes,  si bien la temperatura se puede incluir expl\'icitamente en estas relaciones (v\'ease \citealt{brown2004toward,savage2004effects}) nos centraremos \'unicamente en la masa corporal y dejaremos los efectos de la temperatura para un futuro trabajo. \improvement{agregar un cuadro de los valores de constantes usadas}

\subsubsection{Par\'ametros intra-poblacionales}
Usando las relaciones definidas en \cite{savage2004effects} tenemos:
\begin{equation}\label{eq:params3}
\begin{aligned}
&r = r_0m_\R^{\beta_\R - 1} \\
&q_1=q_{0,1}m_\C^{\beta_\C - 1} \\
&q_2= q_{0,2}m_\PP^{\beta_\PP -1}\\
&K= K_{0}m_\R^{1-\beta_\R}
\end{aligned}
\end{equation}

$\beta_J$ es el exponente que describe la variaci\'on de la tasa metab\'olica a nivel de individuo con la masa de la especie $J$, cuyo valor ha sido descrito entre $2/3 - 1$ siendo $3/4$ el asociado a una tasa metab\'olica basal y valores superiores , a tasas metab\'olicas en actividad\citep{pawar2012dimensionality,west1997general,savage2004predominance}. $J = P,C,R$ \\
$r_0,q_{0,1},q_{0,2}$ y $K_0$ son constantes que dependen de la temperatura y la clase metab\'olica de las especies( e.g. endotermos o ectotermos) , $K_0$ a su ves depende de la productividad del ambiente\citep{pawar2012dimensionality}.
\subsubsection{Par\'amatros inter-poblacionales}
Usando las relaciones definidas en \citep{pawar2012dimensionality,kiltie2000scaling,mcgill2006allometric,bejan2006unifying} tenemos una relaci\'on para $\alpha$  la tasa p\'ercapita de b\'usqueda(tasa de encuentro potencial) y captura de biomasa de recurso $j$ por parte de un depredador $i$, la cual depende del \'area buscada $S_A$ y la tasa de \'exito en la captura $\aleph$ :  \improvement{agregar figuras,elegir mejores palabras}

\begin{equation}\label{eq:alfa}
\begin{aligned}
 \alpha & =  S_A*\aleph \\
 S_A &=  A_D * v_r \\
A_D &=  \begin{cases} 2d & \text{si } D_j = 2 \\ \pi d^2 & \text{si } D_j = 3 \end{cases}\\
v_r &= \sqrt{v_i^2 +v_j^2}\\
\aleph &= \Pi(k_{ji})
\end{aligned}
\end{equation}
A continuaci\'on detallamos cada componente de \eqref{eq:alfa}:\\

$A_D$ se deriva del hecho que el la regi\'on de detecci\'on del depredador $i$ es una ($D_j-1$)-esfera independientemente de la forma de detecci\'on\citep{pawar2012dimensionality}.Donde $D_i$ es la dimensi\'on la cual se desarrolla la interacci\'on,la dimensi\'on del espacio en el cual se distribuye el recurso $i$.
$v_r$ es la velocidad relativa presente entre el $i$ y $j$ , que se interpreta como el promedio poblacional de la rapidez con la cual convergen $i$ y $j$ en el h\'abitat \citep[supinfo.]{pawar2012dimensionality} , la forma dada en \eqref{eq:alfa} asume un movimiento aleatorio por parte de ambas especies, el cual ha sido descrito previamente \citep{okubo2001diffusion}.\\
En \cite{pawar2012dimensionality} describen la variaci\'on de la velocidad para una especie $v$ respecto a cambios en su masa corporal de la siguiente manera:
\begin{equation}\label{eq:vel}
\begin{aligned}
&v \propto \frac{B_0 m^\beta}{Fuerza}\\
&Fuerza \propto m^{\beta_F} \\
&v = v_0m^{\beta - \beta_F}
\end{aligned}
\end{equation}
$v_0$ es una constante que depende del taxa y del modo de locomoci\'on y la constante y la constante metab\'olica $B_0$.\\ \improvement{mejorar que poner, considerar referir anexos}

$k_{ji}= m_j/m_i$ es el raz\'on de la masa del recurso con respecto al depredador , y $\Pi \in [0,1]$ y  puede tomar distintas formas\citep{weitz2006size}, en este trabajo se exploraron las siguientes(figura \ref{fig:efficiency}): 
\begin{equation}\label{eq:sr}
\Pi(k_{ji}) =
\begin{cases}
a\\
\frac{a}{1+k_{ji}^\phi} \\
\end{cases}
\end{equation}

$a \in [0,1] $ es una constante que determina la m\'axima eficiencia en la captura y $\phi > 0 $ determina la forma del decaimiento de $\Pi$ con respecto a $k_{ij}$, note que $Pi \to 0$ , si $k_{ij} \to \infty$ y $Pi \to a $, si $k_{ij} \to 0$.

\begin{figure}
\begin{center}
 \includegraphics[width=0.7\textwidth]{captefficiency.pdf}
 \caption[formas para $\aleph$]{Las tres formas usadas para describir la relaci\'on entre $\aleph$ y $k_{ji}$}
 \label{fig:efficiency} 
\end{center}
\end{figure}

$d$ es el radio de detecci\'on m\'aximo a la cual un depredador de un tama\~no percibe a la presa, la relaci\'on con el tama\~no corporal se desprende de:\\ Sea $\eta$ la agudeza visual del depredador $j$ y $\theta$ el \'angulo de resoluci\'on \improvement{agregar figura}:
\begin{equation}\label{eq:d}
\begin{aligned}
\tan{\theta/2} &= \frac{L_{j}}{d} \\
\theta &= \frac{1}{\eta} \\
\eta & = c_a L_i^{b_a} \\
M &= c_{lm}L^{b_{lm}} 
\end{aligned}
\end{equation}
\subsubsection*{Donde:}
$L_i , L_j$ son las longitudes corporales del recurso $i$ y depredador $j$ respectivamente. \\
$b_{lm} \approx 3 $ y $c_{lm}$ es una constante que depende de la forma y densidad de la especie \citep{peters1986ecological,mcgill2006allometric}. \\
$b_a \approx 1 $ y $c_a$ varia dependiendo del grupo taxon\'omico y el ambiente \citep{kiltie2000scaling} .\\
\cite{pawar2012dimensionality} derivan $d$ de la siguiente relaci\'on:
\[ d = d_0(m_im_j)^{p_d} \]
La cual es una aproximaci\'on plausible derivada de \eqref{eq:d} y es la que nostros usamos.\\
Realizando algunas simplificaciones en \eqref{eq:alfa} se llega a obtener la siguiente forma para $\alpha$ \improvement{referir anexos}
\begin{equation}\label{eq:p4}
\begin{aligned}
&\alpha_1= \alpha_{0,1}m_\C^{p_v+2p_d(D_R-1)}f(k_{\RC})\\
&\alpha_2= \alpha_{0,2}m_\PP^{p_v+2p_d(D_R-1)}f(k_{\RP})\\
&\alpha_3= \alpha_{0,3} m_\PP^{p_v+2p_d(D_C-1)}f(k_{\CP})\\
\end{aligned}
\end{equation}
\subsubsection*{Donde:}
$p_v$ y $p_d$ son exponentes que controlan como escala la velocidad de un individuo  y la distancia de reacci\'on con el tama\~no corporal, respectivamente. El valor de $p_d$ var\'ia ligeramente con la dimensi\'on en la cual se desarrolla la interacci\'on \citep{pawar2012dimensionality}.\\
La funci\'on $f$ incorpora tanto los cambios sobre $S_A$ como $\aleph$ respecto a $k$ y a su vez es dependiente de la estrategia de forrajeo(Fm) del depredador donde tres casos son considerados\citep{pawar2012dimensionality}.
\begin{itemize}
\item Captura activa ($Ac$).
\item Pastoreo ($Gr$).
\item Captura pasiva-\textit{Sit and wait}.($Sw$) 
\end{itemize}

\begin{equation}\label{eq:fkr}
f(k_{ji}) = 
\begin{cases}
\sqrt{1+k_{ji}^{2p_v}}k_{ji}^{(D_j-1)p_d} \Pi(k_{ji}) & Fm = Ac\\
k_{ji}^{p_v+(D_j-1)p_d}\Pi(k_{ji}) & Fm =Gr\\
k_{ji}^{(D_j-1)p_d}\Pi(k_{ji}) & Fm = Sw\\
\end{cases}
\end{equation}

Respecto al tiempo de manejo $t_h$ usamos la relaci\'on derivada en \cite{pawar2012dimensionality} :
\begin{equation}\label{eq:th}
 t_h = h_0m^{-\beta}
\end{equation}
Usando \eqref{eq:th} tenemos:
\begin{equation}
\begin{aligned}
&t_{h_\C} = h_{\C,0} m_\C ^{-\beta_\C} \\
&t_{h_\PP} = h_{\PP,0}m_\PP^{-\beta_\PP}
\end{aligned}
\end{equation}


\subsection{An\'alisis}
Todos los an\'alisis descritos a continuaci\'on se desarrollar\'an para ambos modelos y con distintas combinaciones de dimensi\'on del habitat, estrategia de forrajeo,nivel de productividad ambiental basal y $\Pi$.\\
Dada la parametrizaci\'on empleada el espacio param\'etrico se reduce a tres ejes : $k_\RC,k_\CP, m_p$(note que $k_\RP = k_\RC k_\CP$), se analizar\'an dos casos:
\begin{enumerate}
\item Supondremos que $k_\RC = k_\CP$, lo cual ha sido sugerido previamente \citep{peters1986ecological,brown2004toward}.\label{equalsizeratios}
\item La suposici\'on anterior se relajar\'a y se analizar\'an ciertos valores de masa de depredador $m_p$
\end{enumerate}

\improvement{agregar un diagrama de todas las combinacione usadas}

\subsubsection{Zonas de Coexistencia}
Se delimitaron zonas dentro del espacio param\'etrico donde el triple equilibrio $\mathbf{X} = (R^*,C^*,P^*)$ fuese positivo, $\mathbf{X}$ tiene la propiedad de que $N(\mathbf{X}) = 0$. Para ambos modelos lo que se obtiene es :
\begin{equation}\label{eq:Equilibrio}
D:= \{ (R,C,P) \in \mathbf{R}^3_+ / N((R,C,P)) = 0 \}
\end{equation}
La forma de $D$ es particular de cada modelo.

Las expresiones para el equilibrio son:
\begin{flalign}
R^* &= \frac{K(\epsilon_3 m_C ( \alpha_2 q_1 + \alpha_3 r) - \alpha_1 m_P q_2)}{D}& \\
C^* &= \frac{K\alpha_1 \alpha_2 \epsilon_1 m_P q_2 - K \alpha_2 \epsilon_2 m_C ( \alpha_2 q_1 + \alpha_3 r) + \alpha_3 m_C m_P q_2 r} {\alpha_3 D} \\
P^* &= \frac{m_P(K \alpha_1 m_C(\alpha_2 \epsilon_2 q_1 + \alpha_3 \epsilon_1 \epsilon_3 r) - (K \alpha_1^2 \epsilon_1 m_P q_2 + \alpha_3 \epsilon_3 m_C^2 q_1 r)}{\alpha_3 m_C D }
\end{flalign}
Donde:
\begin{equation}
D = K \alpha_1 \alpha_2 (\epsilon_1 \epsilon_3 - \epsilon_2 ) + \alpha_3 \epsilon_3 m_C r
\end{equation}



\subsubsection{Estabilidad Din\'amica}
En general,podemos determinar la estabilidad local de un punto de equilibrio analizando el valor de los autovalores de la versi\'on linearizada del sistema \eqref{eq:Gsystem}. \citep{yodzis1989introduction}
Usemos la siguiente notaci\'on : $ \frac{\partial F_i}{J} = \partial F_ij $

\begin{equation} \label{eq:linver}
A = \begin{pmatrix}
\left. F_{1R} \right|_{x=\mathbf{X}}& \left.F_{1C}\right|_{x=\mathbf{X}}&\left.F_{1P}\right|_{x=\mathbf{X}}\\
\left. F_{2R}\right|_{x=\mathbf{X}}& \left.F_{2C}\right|_{x=\mathbf{X}}&\left.F_{2P}\right|_{x=\mathbf{X}}\\
\left. F_{3R}\right|_{x=\mathbf{X}}& \left.F_{3C}\right|_{x=\mathbf{X}}&\left.F_{3P}\right|_{x=\mathbf{X}}\\
\end{pmatrix}
\end{equation}

El Polinomio caracter\'istico $P(t)$ cuyas ra\'ices $\lambda$ son los autovalores de $A$ es :

\begin{equation}
\begin{aligned}
& Sea \ F^*_{1J} = \left. F_{1J}\right|_{x=\mathbf{X}} \\
&P(t) = det(A-tI) = - t^3 + a_1t^2 + a_2 t + a3 \\
& a_1 = tr(A) = F^*_{1R}  + F^*_{2C} + F^*_{3P} \\
& a_2 =  -(F^*_{1R}(F^*_{2C}+F^*_{3P}) + F^*_{2C}F^*_{3P} - F^*_{3C}F^*_{2P} - F^*_{1P}F^*_{3R} + F^*_{1C}F^*_{2R}) \\
& a_3 = det(A) 
\end{aligned}
\end{equation}
El sistema se considera localmente estable \citep{yodzis1989introduction} si :
\begin{equation}\label{eq:estab}
\Re(\lambda) < 0 , \forall  \lambda
\end{equation}

Por tanto en cada regi\'on de coexistencia se distinguieron entre puntos estables e inestables  y se determin\'o :
\begin{equation}\label{eq:estabreg}
D_{estab} = \{ (R,C,P) \in D / (R,C,P) \mbox{ es localmente estable} \}
\end{equation}

\subsubsection{Criterios de Invasibilidad}
Se delimitaron zonas dentro del espacio param\'etrico donde era posible la invasi\'on de una de las especies sobre un sistema receptor(debido a esto se les denomina \emph{criterios de invasibilidad}), para el sistema que estamos analizando los siguientes escenarios son posibles: \improvement{agregar figuras de cada escenario}
\begin{itemize}
\item $R$ invade un sistema \emph{vac\'io}.
\item $C$ invade un sistema conformado solo por $R$.
\item $P$ invade un sistema conformado solo por $R$.
\item $P$ invade un sistema conformado por $R$ y $C$.
\item $C$ invade un sistema conformado por $R$ y $P$.
\end{itemize}
En los 3 primeros escenarios la variaci\'on de la Longitud de la cadena tr\'ofica involucra al mecanismo de \textit{adici\'on} y en el \'ultimo escenario el mecanismo involucrado es el de \textit{inserci\'on}.\\

La derivaci\'on de estos criterios asume lo siguiente:
\begin{itemize}
\item \emph{El sistema receptor se ha encontrado aislado por suficiente tiempo como para alcanzar un estado asimpt\'otico y que dicho estado es un punto de equilibrio}.\\Esta suposici\'on es plausible ya que se espera que los eventos de inmigraci\'on de $C$ y $P$ no coincidan y que la separaci\'on entre ambos sea relativamente larga.
\item \emph{La invasi\'on se considera exitosa si es que el crecimiento inicial del invasor es positivo}
\end{itemize}

\subsubsection{R}
En general el sistema se reduce a :
\begin{equation}
\dot{R}= F 
\end{equation}
Para ambos modelos tenemos :
\begin{equation}
\dot{R}= rR(1-R/K)
\end{equation}

Por lo tanto el criterio de invasibilidad para R , $\mathbf{IC_\R}$ es:

\begin{equation}\label{eq:ICR}
\mathbf{IC_\R} := \dot{R} > 0 \iff r > 0
\end{equation}


\subsubsection{C $\to$ R}
En general:
\begin{equation}
\begin{aligned}
\dot{R} &= F - G \\
\dot{C} &= \epsilon_1 G - q_1
\end{aligned}
\end{equation}
            
\myparagraph{caso LV}
\begin{equation}
\begin{aligned}
\dot{R} &= R\left[r(1-R/K)- (\alpha_{1}/m_C) C \right] \\
\dot{C} &= C \left[\epsilon_1 (\alpha_{1}/m_C) R  -q_1 \right]
\end{aligned}
\end{equation}

\begin{equation} \mathbf{IC_{\C \to \R}} := \dot{C} >0 \iff \epsilon_1(\alpha_1/m_C)\hat{R}_1 > q_1  \end{equation}
Donde $\hat{R}_1 = K$ entonces:
\begin{equation} \mathbf{IC_{\C \to \R}} := \epsilon_1(\alpha_1/m_C) K > q_1 \end{equation}
            
\myparagraph{caso RM}
\begin{equation}
\begin{aligned}
\dot{R} &= R\left[r(1-R/K)- \frac{\hat{\alpha_{1}} C}{1+t_{h_{c}} \hat{\alpha_{1}} R} \right] \\
\dot{C} &= C \left[\frac{\epsilon_1 \hat{\alpha_{1}} R}{1+t_{h_{c}}\hat{\alpha_{1}} R} - q_1 \right]
\end{aligned}
\end{equation}
\begin{equation} \mathbf{IC_{\C \to \R}} := \dot{C} > 0 \iff \epsilon_1 \hat{\alpha_{1}} \hat{R}_1 > q_1(1+t_{h_\C}\alpha_{1}\hat{R}_1) \end{equation}
Donde $\hat{R}_1 = K$ entonces:
\begin{equation} \mathbf{IC_{\C \to \R}} := \dot{C} > 0 \iff \hat{\alpha_{1}} K (\epsilon_1 -m_Cq_1 t_{h_\C}) > q_1 \end{equation}

Dada la paremetrizaci\'on usada esto se reduce a:

\begin{equation} \mathbf{IC_{\C \to \R}} := \dot{C} > 0 \iff \hat{\alpha_{1}} K (\epsilon_1 -q_{1,0} h_{\C,0}) > q_1 \end{equation}

\subsubsection{P $\to$ R}
Condiciones similares al caso anterior.
\myparagraph{caso LV}
\begin{equation} \mathbf{IC_{\PP \to \R}} := \dot{P}>0 \iff \epsilon_2(\alpha_2/m_P) K > q_2 \end{equation}
            
\myparagraph{caso RM}\label{r-m-model}

\begin{equation} \mathbf{IC_{\PP \to \R}} := \dot{C} > 0 \iff \hat{\alpha_{2}} K (\epsilon_2 -q_{2,0}h_{\PP,0}) > q_2 \end{equation}

\subsubsection{P $\to$ C-R}
En este y el subsecuente escenario, el sistema es igual a \eqref{eq:Gsystem}
\myparagraph{caso LV}
\begin{equation} \mathbf{IC_{\PP \to \C-\R}} := \dot{P} >0 \iff \epsilon_2\alpha_2\hat{R}_2 + \epsilon_3\alpha_3\hat{C}_2 > q_2 m_P \end{equation}

Donde:
\begin{equation}
\begin{aligned}
\hat{R}_2 &= \frac{q_1 m_C}{\epsilon_1 \alpha_1} \\
\hat{C}_2 &=  r(\frac{m_C}{\alpha_1}) \left[ 1 - \frac{q_1 m_C}{\epsilon_1 \alpha_1 K} \right] 
\end{aligned}
\end{equation}

\myparagraph{caso RM}
\begin{equation} \mathbf{IC_{\PP \to \C-\R}} := \dot{P}> 0 \iff \alpha_2\hat{R}_2(\epsilon_2 - q_{2,0}h_{\PP,0}) + \alpha_3\hat{C}_2(\epsilon_3 -q_{2,0}h_{\PP,0})  > q_2 m_P \end{equation}
Donde:
\begin{equation}
\begin{aligned}
\hat{R}_2 & = \frac{q_1}{\hat{\alpha}_1 ( \epsilon_1 - q_{1,0}h_{\C,0})} \\
\hat{C}_2 & = \frac{1+ t_{h_\C} \alpha_1 \hat{R}_2 }{\hat{\alpha}_1} r(1-\frac{\hat{R}_2}{K}) 
\end{aligned}
\end{equation}

\subsubsection{C $\to$ P-R}

\myparagraph{caso LV}
\begin{equation} \mathbf{IC_{\C \to \PP-\R}} := \dot{C}>0 \iff \epsilon_1(\alpha_1/m_C)\hat{R} - (\alpha_3/m_P)\hat{P}> q_1 \end{equation}
Donde:
\begin{equation}
\begin{aligned}
\hat{R} & = \frac{q_2 m_P}{\epsilon_2 \alpha_2} \\
\hat{P} & = r(\frac{m_P}{\alpha_2}) \left[ 1- \frac{q_2 m_P}{\epsilon_2 \alpha_2 K} \right]
\end{aligned}
\end{equation}
            
\myparagraph{caso RM}
\begin{equation} 
\mathbf{IC_{\C \to \PP-\R}} := \dot{C}>0 \iff \epsilon_1\hat{\alpha}_1\hat{R}_3(1 + t_{h_\PP} \alpha_2 \hat{R}_3) - \hat{\alpha_3}\hat{P}_3(1+t_{h_\C}\alpha_1 \hat{R}_3) - q_1(1+t_{h_\C} \alpha_1 \hat{R}_3)(1+t_{h_\PP} \alpha_2 \hat{R}_3) >0 
\end{equation}
Donde:
\begin{equation}
\begin{aligned}
\hat{R}_3 & =  \frac{q_2}{\hat{\alpha}_2 (\epsilon_2 -q_{2,0} h_{\PP,0})} \\
\hat{P}_3 & =  \frac{1+ t_{h_p}\alpha_2\hat{R}_3}{\hat{\alpha}_2} r(1-\frac{\hat{R}_3}{K})
\end{aligned}
\end{equation}
Reemplazando tenemos:
\begin{equation}
\mathbf{IC_{\C \to \PP-\R}} := (\frac{\epsilon_2}{\epsilon_2 - q_{2,0}h_{\PP,0}})(\hat{\alpha}_1\hat{R}_3(\epsilon_1 - q_{1,0} h_{\C,0}) - q_1) - \hat{\alpha_3}\hat{P}_3(1+t_{h_\C}\alpha_1 \hat{R}_3) >0 
\end{equation}



\subsubsection{M\'axima longitud y grado de Omnivorismo}
La Posici\'on tr\'ofica esta definida por \citep{TP2007proximate}:
\begin{equation} TP_j = \sum_{i \in G}^n (1+TP_i) p_i  \end{equation} donde $G$ es el cojunto de todas las presas del depredador $j$ ,$TP_i$ es la posici\'on tr\'ofica de la especie $i$ y $p_i$ es la proporci\'on de la biomasa total ingerida por el depredador $j$ que deriva de la especie $i$ tal que $ \sum_{i \in G} p_i =1 $.
La m\'axima posici\'on tr\'ofica (\textbf{MTP}) es el mayor valor de \textbf{TP} encontrado en el sistema.\\


En nuestro caso calculamos \textbf{MTP} dentro de la zona de coexistencia siguiendo:
\begin{equation} \mathbf{TP}= \frac{2 I_R + 3 I_C}{I_R+I_C} \end{equation}
Donde:
\begin{itemize}
\item $I_R$ = biomasa de Recurso basal ingerida.
\item $I_C$ = biomasa de Consumidor intermedio ingerida.
\end{itemize}
El grado de omnivor\'ismo $\mathbf{O}$ se define como \citep{TP2007proximate}:
\begin{equation} \mathbf{O}= \frac{I_R}{I_R+I_C} \end{equation}

Fuera de la zona de coexistencia, consideramos que \textbf{MTP} toma los valores de 1 y 2 respectivamente, dependiendo si es que alguno de los consumidores puede invadir, la identidad de dicho consumidor depender\'a de: 
\begin{itemize}
\item El subsistema formado por el invasor y el recurso basal es no invasible,ejem: $C$ puede invadir a $R$ y $P$ no puede invadir a $C-R$.
\item El consumidor es el \'unico en poder invadir.
\item Ambos consumidores pueden invadir pero uno de ellos tiene un mayor crecimiento con el recurso.
\end{itemize}
