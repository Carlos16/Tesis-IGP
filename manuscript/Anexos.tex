\subsection{Influencia de $k_{ij}$ sobre $f$}\label{subsec:funcf}

Sin importar la estrategia de forrajeo tenemos que :
\begin{enumerate}[label=(\alph*)]
\item \begin{equation}
  \lim_{k_{ij} \to  0 }f(k_{ij}) = \lim_{k_{ij} \to 0 }\frac{g(k_{ij}) a}{1+k_{ij}^\phi} = 0 
\end{equation}
ya que $ g(k_{ij}) \to (k_{ij} \to 0)$ y $\frac{a}{1+k_{ij}^\phi} < a$  para todo $k_{ij}$

\item $f_{3D} > f_{2D} \iff k_{ij} > 1$ donde $f_{nD}$ representa al valor de $f$ para espacios de b\'usqueda \emph{n-dimensionales}
\end{enumerate}

A continuaci\'on se analizan propiedades de $f$ que a nivel \emph{cuantitativo} difieren entre las distintas estrategias de forrajeo

\subsubsection{Grazing}

\begin{prop}
  $f$ tiene un punto m\'aximo en $\mathbb{R}^+$ si y solo s\'i $\phi > (D_i - 1)p_d$ 
\end{prop}

\begin{proof}
\mbox{}\\
Sea $ h = (D_i-1)p_d$\\
Dado que $f$ es diferenciable y:
  \begin{equation}
    \begin{aligned}
      f'(k_{ij}) &= a \frac{h k_{ij}^{h-1} (1 + k_{ij}^\phi) -\phi k_{ij}^{\phi-1+h}}{(1+k_{ij}^{\phi})^2} \\
            &= a \frac{(h-\phi)k_{ij}^{h-1 + \phi}  + h k_{ij}^{h-1} }{(1+k_{ij}^{\phi})^2}\\
            &= a \frac{(h-\phi)k_{ij}^\phi + h }{(1+k_{ij}^{\phi})^2}\\
     \end{aligned}
  \end{equation}
Luego tenemos que $f'$ posee un \emph{cero} en $\mathbb{R}^+$ si y solo s\'i $ h< \phi$ y en caso afirmativo tenemos que $k^*$ tal que $f(k^*) = 0$ es $(\frac{h}{\phi-h})^{\frac{1}{\phi}}$ adem\'as:
\begin{equation}
  f'(x) :
  \begin{cases}
    < 0 &; k_{ij}  > k^* \\
    > 0 &; k_{ji} < k^*
  \end{cases}
\end{equation}

Lo que indica que $k^*$ es un punto m\'aximo.
\end{proof}

De la proposici\'on anterior tambi\'en vemos la dependencia de $k^*$ en $\phi$, teniendo $h$ fijo en nuestro caso se observa que si $\phi$ esta suficientemente cercano a $h$ , $k^*$ es extremadamente grande, sin embargo se observa un decaim\'iento muy r\'apido y para valores de $\phi \geq 2h$  , $k^*$ se encuentra pr\'oximo a 1(y en realidad converge a 1 para $\phi \to \infty$) , v\'ease figura \ref{fig:kmaxGrazing}. \\

\begin{figure}
\begin{center}
 \includegraphics[width=0.9\textwidth]{./Plots/kmaxGrazing.pdf}
 \caption[$k^*, Grazing$]{$k^*$ en funci\'on a $\phi$ donde se observa la divergencia para valores cercanos a $h$ y la convergencia a 1 para valores elevados de $\phi$, ({\hwplotB}) es para el caso de ambientes de b\'usqueda $3D$ y ({\hwplotR}) $2D$, $h_{3D}$ y $h_{2D}$ denotan los l\'imites inferiores para $\phi$ que permiten la existencia de $k^*$}
 \label{fig:kmaxGrazing} 
\end{center}
\end{figure}



En el caso que $\phi < h$ tenemos que $f$ es mon\'otona creciente. Ambos casos se grafican en la figura \ref{fig:f1Grazing}.

\begin{figure}
\begin{center}
 \includegraphics[width=0.9\textwidth]{./Plots/f1Grazing.pdf}
 \caption[$f_1, Grazing$]{$f$ en funci\'on a $k_{ij}$, con $a =1$, en el panel de la izquierda tenemos $b = 0.1$ y en el de la derecha $b=1.$ , ({\hwplotB}) es para el caso de ambientes de b\'usqueda $3D$ y ({\hwplotR}) $2D$}
 \label{fig:f1Grazing} 
\end{center}
\end{figure}


\subsubsection{Sit}

Se tiene cualitativamente las m\'ismas caracter\'isticas que en el caso anterior, con la diferencia que en este caso $ h:= p_v + 2(D_i -1) p_d$, y por ende para $\phi \in ]2(D_i - 1) p_d , p_V + 2(D_i -1)p_d[$ tendr\'iamos un comportamiento mon\'ontono para $f_{sit}$ y la existencia de un m\'aximo para $f_{grazing}$.\\

La figura \ref{fig:f1Sit} muestra las semejanzas con el caso anterior, salvo la diferencia que en este caso $k^*>1$ para el caso $3D$ y adem\'as alcanza un valor m\'as alto que el caso $2D$.\\
En general tenemos que $ k^*_{Sit} > k^*_{grazing}$ .

\begin{figure}
\begin{center}
 \includegraphics[width=0.9\textwidth]{./Plots/f1Sit.pdf}
 \caption[$f_1, Sit$]{$f$ en funci\'on a $k_{ij}$, con $a =1$, para el caso de una estrategia de forrajeo \emph{Sit-and-Wait}, las dem\'as especificaciones se comparten con la figura \ref{fig:f1Grazing}}
 \label{fig:f1Sit} 
\end{center}
\end{figure}


\subsubsection{Active}

En este caso tenemos que :
\begin{equation}
  \lim_{k_{ij} \to 0 } f_{active}(k_{ij}) - f_{grazing}(k_{ij}) = 0 \ \ \land \lim_{k_{ij} \to \infty}f_{active}(k_{ij}) - f_{sit}(k_{ij}) = 0
\end{equation}

Por lo tanto en ambos extremos podemos esperar un comportamiento similar al descrito en los dos casos anteriores, esto es para $\phi > p_v + 2(D_i -1) p_d$ tenemos que $f$ decae exponencialmente a partir de un valor de $k_{ij}$ \emph{suficientemente grande}, y a su vez crece exponencialmente para valores \emph{suficientemente peque\~nos}, dado que $f \in C^1$ esto a su vez nos dice que debe existir un punto m\'aximo para $f$ sin embargo en este caso no tenemos una expresi\'on an\'alitica para $k^*$ salvo que cumple la siguiente relaci\'on:

\begin{equation}
  (k^*)^{\phi}((k^*)^{2p_v}(p_v+ h -\phi) + (h -\phi) + (k^*)^{2p_v - \phi}(p_v + h ) ) + h = 0
\end{equation}

Con $h$ igual que en el caso $grazing$.\\
De esta relaci\'on podemos obtener que :
\begin{equation}
  k^*_{Active} \in ] k^*_{Grazing} , k^*_{Sit} [
\end{equation}
Los detalles se dejan al lector como ejercicio.\\

A su vez para $\phi \leq 2(D_i - 1)p_d $ podemos esperar un crecimiento mon\'otono por parte de $f$. 

\begin{figure}
\begin{center}
 \includegraphics[width=0.9\textwidth]{./Plots/f1Active.pdf}
 \caption[$f_1, Active$]{$f$ en funci\'on a $k_{ij}$, con $a =1$, para el caso de una estrategia de forrajeo \emph{activa}, las dem\'as especificaciones se comparten con la figura \ref{fig:f1Grazing}}
 \label{fig:f1Active} 
\end{center}
\end{figure}




\subsection{Influencia de las masas sobre estados de equilibrio de las comunidades receptoras}

A continuaci\'ion usamos la notaci\'on empleada en resultados siempre que sea posible.
\subsubsection{R-C}
Tenemos que al equilibrio $R$ y $C$ estan determinados por el siguiente par de expresiones
\begin{equation}
  \begin{aligned}
    R_{eq} &= \frac{q_{0,1} m_C^{\beta-h}}{\varepsilon_1 \alpha_{0,1} f_1(k_\RC)}\\
    C_{eq} &= \frac{r_0 m_C^{\beta -h}}{\alpha_{0,2} k_\RC^{1 - \beta}f_1(k_\RC)}(1 - \frac{R_{eq}}{\kappa_0 (k_\RC m_C)^{1-\beta}})
  \end{aligned}
\end{equation}

Las condiciones de existencia de equilibrio positivo son equivalentes a las condiciones para la invasi\'on de $C$ y se detallan en Resultados.\\
Dentro del espacio param\'etrico que posibilita la coexistencia tenemos asu vez que el efecto que causa cambios en $m_C(k_\CP m_P)$ es dependiente de la dimensi\'on del espacio de b\'usqueda: para espacios $3D$ tiene un impacto negativo sobre $R_{eq}$ y lo contrario ocurre en el caso $2D$. Su influencia sobre $C_{eq}$ es mas dif\'icil de diferenciar , en el caso $2D$ tenemos que el impacto es positivo, sin embargo en $3D$ depende del valor de $m_C$ ya que:
\begin{equation}
  \frac{\partial C_{eq}}{\partial m_C}  = d_0 ( (b-h) + (1 + 2h - 3 \beta )\frac{\chi_1}{\chi_0} m_C^{2\beta -h -1}) 
\end{equation}
Entonces la regi\'on de crecimiento y decrecimiento respecto a $m_C$ esta determinada por:
\begin{equation}
  \begin{cases}
    Crecimiento &  m_C^{1 + h - 2\beta} < \frac{q_{0,1}(1+2h -3 \beta)}{\chi_0(h - \beta)}\\
    Decrecimiento &  m_C^{1 + h - 2\beta} > \frac{q_{0,1}(1+2h -3 \beta)}{\chi_0(h - \beta)}
  \end{cases}
\end{equation}

\begin{figure}
  \centering
  \includegraphics[width = 0.99\textwidth]{./Plots/RCeq.pdf}
  \caption[Equilibrio $R-C$vs$m_C$]{Equilibrios para el subsistema $R-C$ en funci\'on a $m_C$ para $k_\RC = 10^{-2}$, donde se observa las diferencias existentes entre espacios de b\'usqueda $2D$({\hwplotG}) y $3D$ ({\hwplotB}), en el panel de la derecha se representa a su vez el valor de $m_C$ para el cual $C_{eq}$ es m\'aximo.$b = 0.1$,$\kappa_0 = 0.1 , 30$  en espacios $2D$ y $3D$ respectivamente.}
  \label{fig:EqRCmC}
\end{figure}

La influencia de $k_\RC$(para un $m_C$ fijo)sobre los valores de equilibrio es an\'aloga a su influencia sobre la invasibilidad de $C$, dentro de la zona de coexistencia afecta negativamente a $R_{eq}$ en la zona de crecimiento de $f$ y positivamente en la zona de decrecimiento(si existiera), es decir se espera un valor m\'aximo de $R_{eq}$ para valores de $k_\RC$ cercanos a los l\'imites de coexistencia, la influencia sobre $C_{eq}$ al igual que en el caso anterior m\'as complicada teniendo:
\begin{equation}
  \frac{\partial C_{eq}}{\partial k_\RC}  = \frac{e_0 g'(k_\RC) }{g(k_\RC)^2} ( -1  + \frac{2e_1}{g(k_\RC)})
\end{equation}
Donde:
\begin{equation}
  \begin{aligned}
    e_0 &=  \frac{r_0 m_C^{\beta -h}}{\alpha_{0,2}} \\
    e_1 &=  \frac{q_{0,1} m_C^{2\beta-h-1}}{\varepsilon_1 \alpha_{0,1} \kappa_0}
  \end{aligned}
\end{equation}

Por tanto:
\begin{equation}
  \begin{cases}
    Crecimiento &  g'(k_\RC) ( \frac{2e_1}{g(k_\RC)} - 1) > 0 \\
    Decrecimiento &  g'(k_\RC) ( \frac{2e_1}{g(k_\RC)} - 1) < 0
    \end{cases}
\end{equation}

Dado que $g'(k_\RC)$ es positivo para $k_\RC$ peque\~nos y a su vez $g(k_\RC)$ disminuye , podemos esperar que para $k_\RC$ suficientemente peque\~nos $k_\RC$ afecte positivamente a $C_{eq}$ ,por otro lado dependiendo del valor de $\phi$ y la estrategia de forrajeo podemos tener un comportamiento cualitativo diferente para valores intermedios de $k_\RC$, en el caso m\'as simple $g'(k_\RC)$ siempre es positivo y por ende solo existir\'ia un punto de transici\'on entre zona de crecimiento y decrecimiento dado por la condici\'on $g(k_\RC) = 2e_1$ , sin embargo en el caso que $g(k_\RC)$ posea a su vez zonas de crecimiento y decrecimiento, observar\'iamos 3 transiciones $k_\RC^1 , k_\RC^2$ y$k_\RC^3$ , determinadas por $g(k_\RC^1) = 2e_1$ , $g'(k_\RC^2) =0$ y $g(k_\RC^3) = 2e_1$ , esto siempre que $k_\RC^1 < k_\RC ^2 < k_\RC^3$ ,es decir siempre que $g$ crezca los suficiente, en este caso observamos que $\kappa_0$ y $m_C$ favorecer\'ian la existencia de estos puntos debido a su influencia negativa sobre $e_1$.


\begin{figure}
  \centering
  \includegraphics[width = 0.99\textwidth]{./Plots/RCeqKRC.pdf}
  \caption[Equilibrio $R-C$vs$k_\RC$]{Equilibrios para el subsistema $R-C$ en funci\'on a $k_\RC$ para $m_C = 10^{-3}$ , $b = 1.0 $,$\kappa_0 = 0.1 , 30$  en espacios $2D$ ({\hwplotG}) y $3D$ ({\hwplotB}) respectivamente. En el panel de la derecha se observan las tres transiciones entre zonas de crecimiento y decrecimiento, y en ambos casos la presencia de puntos m\'aximos cercanos a los l\'imites de coexistencia.}
  \label{fig:EqRCkRC}
\end{figure}

\subsubsection{R-P}
La descripci\'on es an\'aloga al caso anterior salvo que intercambiamos $m_C$ por $m_P$ y $k_\RC$ por $k_\RP$ en nuestro argumento.

\begin{equation}
  \begin{aligned}
    R_{eq} &= \frac{q_{0,2} m_P^{\beta-h}}{\varepsilon_2 \alpha_{0,2} f_2(k_\RP)}\\
    P_{eq} &= \frac{r_0 m_P^{\beta -h}}{\alpha_{0,2} k_\RP^{1 - \beta}f_2(k_\RP)}(1 - \frac{R_{eq}}{\kappa_0 (k_\RP  m_P)^{1-\beta}})
  \end{aligned}
\end{equation}









\subsection{Derivaci\'on de Criterios y Zonas de Invasibilidad}\label{subsec:CI}

\subsubsection{R}
El sistema se reduce a:
\begin{equation}
\dot{R}= rR(1-R/K)
\end{equation}

Por lo tanto el criterio de invasibilidad para R , $\mathbf{IC_\R}$ es:

\begin{equation}\label{eq:ICR}
\mathbf{IC_\R} := \dot{R} > 0 \iff r > 0
\end{equation}

\begin{equation}
\mathbf{Z(I_{\R})} := \{ v \in \mathbb{R}^3_+ / \dot{R}(v)>0 \}
\end{equation}

\subsubsection{C $\to$ R}
El sistema se reduce a :
\begin{equation}
\begin{aligned}
\dot{R} &= R\left[r(1-R/K)- (\alpha_{1}) C \right] \\
\dot{C} &= C \left[\epsilon_1 (\alpha_{1}) R  -q_1 \right]
\end{aligned}
\end{equation}


\begin{equation} \mathbf{IC_{\C \to \R}} := \dot{C} >0 \iff \epsilon_1(\alpha_1)\hat{R}_1 > q_1  \end{equation}
Donde $\hat{R}_1 = K$ entonces:
\begin{equation} \mathbf{IC_{\C \to \R}} := \epsilon_1(\alpha_1) K > q_1 \end{equation}
            
\begin{equation}
\mathbf{Z(I_{\C \to \R})} := \{v \in Z(I_{\R}) / \dot{C}(v) > 0\}
\end{equation}


\subsubsection{P $\to$ R}
El sistema es similar al caso anterior, intercambiando $P$ por $C$.

\begin{equation} \mathbf{IC_{\PP \to \R}} := \dot{P}>0 \iff \epsilon_2(\alpha_2) K > q_2 \end{equation}

\begin{equation}
\mathbf{Z(I_{\PP \to \R})} := \{v \in Z(I_{\R}) / \dot{P}(v) > 0\}
\end{equation}

            
\subsubsection{P $\to$ C-R}

\begin{equation} \mathbf{IC_{\PP \to \C-\R}} := \dot{P} >0 \iff \epsilon_2\alpha_2\hat{R}_2 + \epsilon_3\alpha_3\hat{C}_2 > q_2 \end{equation}

Donde:
\begin{equation}
\begin{aligned}
\hat{R}_2 &= \frac{q_1}{\epsilon_1 \alpha_1} \\
\hat{C}_2 &=  (\frac{r}{\alpha_1}) \left[ 1 - \frac{q_1}{\epsilon_1 \alpha_1 K} \right] 
\end{aligned}
\end{equation}

\begin{equation}
\mathbf{Z(I_{\PP \to \C-\R})} := \{v \in Z(I_{\C \to \R}) / \dot{P}(v) > 0\}
\end{equation}


\subsubsection{C $\to$ P-R}

\begin{equation} \mathbf{IC_{\C \to \PP-\R}} := \dot{C}>0 \iff \epsilon_1(\alpha_1)\hat{R} - (\alpha_3)\hat{P}> q_1 \end{equation}
Donde:
\begin{equation}
\begin{aligned}
\hat{R} & = \frac{q_2}{\epsilon_2 \alpha_2} \\
\hat{P} & = \frac{r}{\alpha_2}\left[ 1- \frac{q_2}{\epsilon_2 \alpha_2 K} \right]
\end{aligned}
\end{equation}

\begin{equation}
\mathbf{Z(I_{\C \to \PP-\R})} := \{v \in Z(I_{\PP \to \R}) / \dot{C}(v) > 0\}
\end{equation}


\subsection{Calculo de Equilibrio para un modelo Lotka-Volterra}\label{subsec:equil}
El c\'alculo del equilibrio se reduce a la soluci\'on del siguiente sistema linear:
\begin{equation}
\begin{pmatrix}
r/K & \alpha_1 & \alpha_2 \\
(\alpha_1\epsilon_1& 0 & -\alpha_3 \\
\alpha_2 \epsilon_2 & \alpha_3 \epsilon_3 & 0
\end{pmatrix}
\begin{pmatrix}
R^* \\
C^* \\
P^* 
\end{pmatrix}
=
\begin{pmatrix}
r \\
q_1 \\
q_2
\end{pmatrix}
\end{equation}

Para solucionarlo usamos la regla de Kramer, en el caso D = 0 el sistema no presenta soluci\'on no trivial.

\subsection{Estabilidad Din\'amica}\label{subsec:stab}
En general,podemos determinar la estabilidad local de un punto de equilibrio analizando el valor de los autovalores de la versi\'on linearizada del sistema \eqref{eq:Gsystem}. \citep{yodzis1989introduction}
Usemos la siguiente notaci\'on : $ \frac{\partial F_i}{\partial J} = F_{ij} $

\begin{equation} \label{eq:linver}
A = \begin{pmatrix}
\left. F_{1R} \right|_{x=\mathbf{X}}& \left.F_{1C}\right|_{x=\mathbf{X}}&\left.F_{1P}\right|_{x=\mathbf{X}}\\
\left. F_{2R}\right|_{x=\mathbf{X}}& \left.F_{2C}\right|_{x=\mathbf{X}}&\left.F_{2P}\right|_{x=\mathbf{X}}\\
\left. F_{3R}\right|_{x=\mathbf{X}}& \left.F_{3C}\right|_{x=\mathbf{X}}&\left.F_{3P}\right|_{x=\mathbf{X}}\\
\end{pmatrix}
\end{equation}

El Polinomio caracter\'istico $P(t)$ cuyas ra\'ices $\lambda$ son los autovalores de $A$ es :

\begin{equation}
\begin{aligned}
& Sea \ F^*_{1J} = \left. F_{1J}\right|_{x=\mathbf{X}} \\
&P(t) = det(A-tI) = - t^3 + a_1t^2 + a_2 t + a3 \\
& a_1 = tr(A) = F^*_{1R}  + F^*_{2C} + F^*_{3P} \\
& a_2 =  -(F^*_{1R}(F^*_{2C}+F^*_{3P}) + F^*_{2C}F^*_{3P} - F^*_{3C}F^*_{2P} - F^*_{1P}F^*_{3R} + F^*_{1C}F^*_{2R}) \\
& a_3 = det(A) 
\end{aligned}
\end{equation}
El sistema se considera localmente estable \citep{yodzis1989introduction} si :
\begin{equation}\label{eq:estab}
\Re(\lambda) < 0 , \forall  \lambda
\end{equation}

\subsection{Par\'ametros usados}\label{subsec:params}
\begin{table}
\begin{longtable}{|c|c|}
\hline
Par\'ametros & Valores usados \\
\hline
$a$ & 1 \\
$\phi$ & 0.02 , 0.2 , 2 \\
\hline
$\kappa_0$ & 3D : 3 , 30,300 \\
   & 2D : 0.01,0.1,1 \\
\hline
$\beta$ & 0.75\\
\hline
$p_V$ & 0.26 \\
$p_d$ & 0.21 (2D) \\
      & 0.20 (3D) \\
\hline
$\alpha_0$ & $10^{-3.08}$(2D) \\
         & $10^{-1.77}$(3D) \\
\hline
$r_0$ & $1.71 \times 10^{-6}$ \\
\hline
$q_0$ & $4.15 \times 10^{-8}$\\
\hline
& $Ac-Ac-Ac$ \\
Estrategias de forrajeo& $Gr-Gr-Ac$ \\
& $Ac-Sw-Sw$ \\
\hline
\end{longtable}
\end{table}
