\section{Abstract}
The body mass of a species due to its connection with metabolic rate and movement is a key trait which determine a lot of the species functionality, in this work we extend this line of thinking further by exploring how the body mass of the species which compose an intraguild predation module influence the maximum trophic position observed in the system and its structuration path (sequence of \textit{plausible} invasions) and coexistence zone. We derive relations that the mass of the interacting species must fulfill in order for an assembly path to be expressed and for the existence of a positive equilibrium, the form of this relationships is found to be insensitive to changes in basal productivity, foraging dimension and strategy; but they have a quantitative influence and in the case of the latter two, they could affect qualitative behavior of some subparts of the relationship with respect to changes in the predator prey ratios present in the module. By its influence over this processes the combination of body masses affect the food chain length that could be found in the module. This results suggests promising conections between the assembly process of a comunity and the body mass of the species present in the regional pool and receptor comunity.
