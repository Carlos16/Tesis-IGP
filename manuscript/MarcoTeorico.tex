

\subsection{Modelos Matem\'aticos \emph{Depredador-Presa}}

El uso de modelos matem\'aticos para describir la din\'amica poblacional de un conjunto de especies que se relacionan mediante interacciones depredador-presa, se remonta a los trabajos de Lotka y Volterra\citep{gotelliprimer}, los cuales independientemente describieron los cambios en la abundancia poblacional de un depredador $C$ y una presa $R$ mediante el siguiente sistema de ecuaciones diferenciales : 

\begin{equation}\label{eq:LV}
\begin{aligned}
&\dot{R} = R(r - \alpha C)\\
&\dot{C} = C(e\alpha - q) 
\end{aligned}
\end{equation}

El sistema \eqref{eq:LV} es conocido como \emph{sistema de ecuaciones depredador-presa Lotka-Volterra} y se puede extender f\'acilmente a un sistema de $n$ especies identificadas con $\{1,2, \ldots ,n\}$ de la siguiente manera:
\begin{equation}\label{eq:LVGen}
\dot{X_i}  = X_i(b_i - d_i + \sum_{j}^n \alpha_{ij} X_j)
\end{equation}

Donde $X_i$ representa la abundancia en n\'umeros o biomasa de la especie $i$, $b_i$ la tasa intr\'inseca de producci\'on de masa(individuos), $d_i$ la tasa de p\'erdida de masa(individuos) y $\alpha_{ij}$ se puede interpretar como la p\'erdida(ganancia) de masa(individuos) debido a interacciones con las dem\'as especies presentes en el h\'abitat.

Por lo general se asume que :
\begin{itemize}
\item $\alpha_{ij}$ es una constante(la modificaci\'on $\alpha_{ij} = \alpha_{ij}(X)$ para una forma especf\'icia de la funci\'on da lugar a lo que se conoce como respuesta funcional tipo II  o tipo III).
\item $\alpha_{ii} < 0 $ si $X_i$ es un recurso basal, a diferencia de la formulaci\'on original en este caso se asume que existe \emph{denso dependencia directa} entre los individuos de una poblaci\'on de presas.
\item $\alpha_{ii} = 0 $ si $X_i$ es un depredador, es decir no existe \emph{denso dependencia directa} entre individuos de una poblaci\'on de depredadores, sin embargo la denso dependecia se da indirectamente a traves de la interacci\'on con los recursos..
\item $b_i = 0$ si $X_i$ es un depredador, ya que los depredadores no pueden subsistir en ausencia de presas.
\end{itemize}
Modelos como \eqref{eq:LVGen} son llamados \emph{modelos tipo Lotka-Volterra}.

El modelo depredador-presa \emph{Lotka-Volterra},pese a sus evidentes limitaciones,ha jugado(y sigue jugando) un papel importante en el desarrollo de la teor\'ia en ecolog\'ia y form\'o la base para el desarrollo de modelos que incorporan caracter\'isticas con mayor fundamento biol\'ogico. \\
Uno de los problemas que surge al momento de construir un modelo similar al de \eqref{eq:LVGen} es el como decidir \emph{a priori} el valor de los par\'ametros del modelo(e.g. $b_i , d_i$). Este problema se hace m\'as notorio conforme la dimensi\'on del modelo(i.e. n\'umero de ecuaciones en el sistema) crece \citep{yodzis1992body}.\\
Yodzis e Innes en su seminal paper \emph{Body size and Consumer-Resource dynamics}\citep{yodzis1992body} propusieron una forma para aligerar el problema.Ellos introdujeron lo que hoy en d\'ia se conoce como \emph{modelamiento bioenerg\'etico} el cual se basa en derivar los valores de los par\'ametros de las relaciones que existen entre ellos y la masa corporal, esta relaci\'on es generalmente de forma indirecta y se manifiesta debido a la influencia que tiene la masa sobre el metabolismo de las especies\citep{peters1986ecological}. A la fecha se han desarrollado diversos refinamientos a estas ideas, lo cual nos permite centrarnos en par\'ametros con mayor significado biol\'ogico.\citep{kiltie2000scaling,brown2004toward,savage2004predominance,pawar2012dimensionality,brose2010body}

\subsection{Ensamblaje}
En esta secci\'on definimos ciertos t\'erminos relacionados al proceso de ensamblaje de una comunidad. \\
Se denomina proceso de ensamblaje $E_A$ de una comunidad $A$ asociada a un h\'abitat $H$ al continuo de colonizaciones y extinciones de especies que se dan dentro de $H$, el conjunto de especies que \emph{potencialmente} puede colonizar a la comunidad $A$(i.e pueden por lo menos llegar a $H$) se denomina \emph{pool regional de especies}. 
\begin{equation}\label{eq:Assembly}
E_A := (A_0,H_0) \to (A_1,H_1) \to (A_2,H_2) \to \ldots
\end{equation}
En \eqref{eq:Assembly} cada cambio de $(A_i,H_i)$ a $(A_{i+1},H_{i+1})$ se da debido a un intento de colonizaci\'on sobre $(A_i,H_i)$,el cual puede tener uno de los tres siguientes desenlaces \citep{pawar2009community}: 
\begin{enumerate}
\item El invasor no puede invadir.
\item El invasor llega a invadir pero provoca extinciones en la comunidad receptora, pudiendo el mismo exintiguirse, esto es llamado \emph{invasi\'on inestable}.
\item El invasor llega a invadir y no provoca ninguna extinci\'on, esto es llamado \emph{invasi\'on estable}.
\end{enumerate}

De lo anterior se observa que la comunidad y h\'abitat f\'isico receptor juega un papel muy importante en el subsiguiente paso del proceso, adem\'as especificamos que cada estado $(A_i,H_i)$ puede estar en constante cambio,excepto por el n\'umero de especies, debido a la din\'amica inherente de las poblaciones de las especies presentes y por ende el tiempo entre distintas colonizaciones asu vez puede influenciar el desenlace de la colonizaci\'on debido a que la comunidad receptora puede estar en distintos estados(e.g. estado transiente $vs$ estado asint\'otico).\\

Sea $B = \{ (A_j,H_j) / j \in I, I \subseteq \mathbb{N}\}$ un conjunto de estados en $E_A$ , usando el formalismo definido en \eqref{eq:Assembly} decimos que la comunidad $A$ \emph{$\omega$-converge} a $B$ durante el ensamblaje si existe un $ m \in \mathbb{N}$ tal que $(A_i,H_i) \in B, i \geq m$. Sea $D$ el menor de dichos conjuntos. Decimos entonces que la comunidad $A$ \emph{converge} a $D$.

\subsubsection{Caminos de Ensamblaje}
Sea $\hat{A}$ un conjunto de especies con un set de interacciones $S$ , una secuencia de colonizaciones $E$ tal que el estado final es $(\hat{A},S)$ es llamado \emph{camino de ensamblaje} hacia $(\hat{A},S)$.

\subsection{ M\'odulo IGP}

En teor\'ia de redes tr\'oficas, se denomina \emph{community modules} a un conjunto peque\~no de especies que interact\'uan entre s\'i y cuyo patr\'on de interacci\'on ha sido encontrado en diversos ecosistemas(citar holt). Entre ellos el m\'odulo de \emph{depredaci\'on dentro del clan}(IGP por sus siglas en ingl\'es) fue descrito por primera vez por \cite{polis1989ecology} envuelve tres especies: un recurso basal, un depredador intermedio y un depredador tope.Ambos depredadores consumen al recurso y ademas el depredador intermedio es consumido por el depredador tope. Este sistema pese a tener solo $3$ especies incorpora diversos tipos de interacci\'on \emph{depredaci\'on, competencia aparente, competencia por explotaci\'on y mutualismo indirecto}. Esto se describe gr\'aficamente en la figura ~\ref{fig:IGP}
\begin{figure}[h]
\begin{tikzpicture}[<->,>=stealth',shorten >=1pt,auto,
  thick,main node/.style={circle,fill=blue!20,draw,font=\sffamily\Large\bfseries}]


\node[main node] (R) at (0,-3){R};
\node[main node] (C) at (3,0) {C};
\node[main node] (P) at (0,3) {P};

 \path[every node/.style={font=\sffamily\small}]

(R) edge [<-,red] (C)
(R) edge [<-,red] (P)
(P) edge [->,red] (C)
(P) edge [bend right , dotted, blue] (R) 
(P) edge [bend left , dotted, green]  (C)
(R) edge [bend right ,dotted , black] (C);



\begin{customlegend}[legend cell align=left,
legend entries={ % <= in the following there are the entries
Depredaci\'on,
Competencia por explotaci\'on,
Competencia aparente, 
Mutualismo indirecto
},
legend style={at={(-1.5,3.5)},font=\footnotesize}] % <= to define position and font legend
% the following are the "images" and numbers in the legend
    \addlegendimage{-stealth,red}
    \addlegendimage{stealth-stealth,dotted,green}
    \addlegendimage{stealth-stealth,dotted,black}
    \addlegendimage{stealth-stealth,dotted,blue}
    
\end{customlegend}

\end{tikzpicture}

\caption{M\'odulo \textbf{IGP}}
\label{fig:IGP}
\end{figure}

Adem\'as este sistema es el menor(en el sentido de n\'umero de especies) en el cual podemos encontrar 2 \emph{caminos de ensamblaje}.
\begin{figure}[h]
\begin{center}
\begin{tikzpicture}[->,>=stealth',shorten >=1pt,auto,
  thick,main node/.style={circle,fill=blue!20,draw,font=\sffamily\Large\bfseries}]

\node[main node](R1) at (1,0) { R};
\node[main node](C1) at (0,2) {C};
\node[main node](C2) at  (4,2){C};
\node[main node](R2) at ($(R1) + (3,0)$) {R};
\node[main node](P1) at (5,4){P};
\node[main node](R3) at ($(R2) + (3,0)$){R};
\node[main node](C3) at ($(R3) + (1,2)$){C};
\node[main node](P3) at ($(R3) + (0,4)$){P};
\node[font = \fontsize{15}{15}\selectfont](A1) at (-4,0) {Camino 1};

\path
(C1) edge[bend left,dotted] (R1)
($(R1)+(1.0,0)$) edge[line width = 2]  ($(R1)+ (2,0)$)
(R2) edge (C2)
(P1) edge[bend right,dotted] (C2)
(P1) edge[bend left,dotted] (R2)
($(R2) + (1.0,0)$) edge[line width  = 2] ($(R2) + (2,0)$)
(R3) edge (C3)
(R3) edge (P3)
(C3) edge (P3);
\end{tikzpicture}
$\left. \right.$\\
$\left. \right.$\\

\begin{tikzpicture}[->,>=stealth',shorten >=1pt,auto,
  thick,main node/.style={circle,fill=blue!20,draw,font=\sffamily\Large\bfseries}]

\node[main node](R1) at (1,0) { R};
\node[main node](P1) at (0,2) {P};
\node[main node](P2) at  (4,4){P};
\node[main node](R2) at ($(R1) + (3,0)$) {R};
\node[main node](C1) at (5,2){C};
\node[main node](R3) at ($(R2) + (3,0)$){R};
\node[main node](C3) at ($(R3) + (1,2)$){C};
\node[main node](P3) at ($(R3) + (0,4)$){P};
\node[font = \fontsize{15}{15}\selectfont](A1) at (-4,0) {Camino 2};

\path
(P1) edge[bend left,dotted] (R1)
($(R1)+(1.0,0)$) edge[line width = 2]  ($(R1)+ (2,0)$)
(R2) edge (P2)
(C1) edge[bend left,dotted] (R2)
(P2) edge[bend left,dotted,red] (C1)
($(R2) + (1.0,0)$) edge[line width  = 2] ($(R2) + (2,0)$)
(R3) edge (C3)
(R3) edge (P3)
(C3) edge (P3);

\end{tikzpicture}
\end{center}

\caption{\emph{Caminos de ensamblaje} para el m\'odulo \textbf{IGP}}
\label{fig:IGPAssembly}
\end{figure}
Numerosos ejemplos son citados en \cite{polis1989ecology}, algunos de los cuales se presentan en la siguiente tabla:
\improvement{HACER LA TABLA!}

\subsection{Cadenas Tr\'oficas}


