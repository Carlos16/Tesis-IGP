

\subsection{Modelos Matem\'aticos \emph{Depredador-Presa}}

El uso de modelos matem\'aticos para describir la din\'amica poblacional de un conjunto de especies que se relacionan mediante interacciones depredador-presa, se remonta a los trabajos de Lotka y Volterra\citep{gotelliprimer}, los cuales independientemente describieron los cambios en la abundancia poblacional de un depredador $C$ y una presa $R$ mediante el siguiente sistema de ecuaciones diferenciales : 

\begin{equation}\label{eq:LV}
\begin{aligned}
&\dot{R} = R(r - \alpha C)\\
&\dot{C} = C(e\alpha - q) 
\end{aligned}
\end{equation}

El sistema \eqref{eq:LV} es conocido como \emph{sistema de ecuaciones depredador-presa Lotka-Volterra} y se puede extender f\'acilmente a un sistema de $n$ especies identificadas con $\{1,2, \ldots ,n\}$ de la siguiente manera:
\begin{equation}\label{eq:LVGen}
\dot{X_i}  = X_i(b_i - d_i + \sum_{j}^n \alpha_{ij} X_j)
\end{equation}

Donde $X_i$ representa la abundancia en n\'umeros o biomasa de la especie $i$, $b_i$ la tasa intr\'inseca de producci\'on de masa(individuos), $d_i$ la tasa de p\'erdida de masa(individuos) y $\alpha_{ij}$ se puede interpretar como la p\'erdida(ganancia) de masa(individuos) debido a interacciones con las dem\'as especies presentes en el h\'abitat.

Por lo general se asume que :
\begin{itemize}
\item $\alpha_{ij}$ es una constante.
\item $\alpha_{ii} < 0 $ si $X_i$ es un recurso basal, a diferencia de la formulaci\'on original en este caso se asume que existe \emph{denso dependencia instant\'anea} entre los individuos de una poblaci\'on de presas.
\item $\alpha_{ii} = 0 $ si $X_i$ es un depredador, es decir no existe \emph{denso dependencia instan\'anea} entre individuos de una poblaci\'on de depredadores.
\item $b_i = 0$ si $X_i$ es un depredador, ya que los depredadores no pueden subsistir en ausencia de presas.
\end{itemize}
Modelos como \eqref{eq:LVGen} son llamados \emph{modelos tipo Lotka-Volterra}.


El modelo depredador-presa \emph{Lotka-Volterra},pese a sus evidentes limitaciones,ha jugado(y sigue jugando) un papel importante en el desarrollo de la teor\'ia en ecolog\'ia y form\'o la base para el desarrollo de modelos que incorporan caracter\'isticas con mayor fundamento biol\'ogico.


\subsection{ M\'odulo IGP}

En teor\'ia de redes tr\'oficas, se denomina \emph{community modules} a un conjunto peque\~no de especies que interact\'uan entre s\'i y cuyo patr\'on de interacci\'on ha sido encontrado en diversos ecosistemas(citar holt). Entre ellos el m\'odulo de \emph{depredaci\'on dentro del clan}(IGP por sus siglas en ingl\'es) fue descrito por primera vez por \cite{polis1989ecology} envuelve tres especies: un recurso basal, un depredador intermedio y un depredador tope.Ambos depredadores consumen al recurso y ademas el depredador intermedio es consumido por el depredador tope. Este sistema pese a tener solo $3$ especies incorpora diversos tipos de interacci\'on \emph{depredaci\'on, competencia aparente, competencia por explotaci\'on y mutualismo indirecto}. Esto se describe gr\'aficamente en la figura ~\ref{fig:IGP}
\begin{figure}[h]
\begin{tikzpicture}[<->,>=stealth',shorten >=1pt,auto,
  thick,main node/.style={circle,fill=blue!20,draw,font=\sffamily\Large\bfseries}]


\node[main node] (R) at (0,-3){R};
\node[main node] (C) at (3,0) {C};
\node[main node] (P) at (0,3) {P};

 \path[every node/.style={font=\sffamily\small}]

(R) edge [<-,red] (C)
(R) edge [<-,red] (P)
(P) edge [->,red] (C)
(P) edge [bend right , dotted, blue] (R) 
(P) edge [bend left , dotted, green]  (C)
(R) edge [bend right ,dotted , black] (C);



\begin{customlegend}[legend cell align=left,
legend entries={ % <= in the following there are the entries
Depredaci\'on,
Competencia por explotaci\'on,
Competencia aparente, 
Mutualismo indirecto
},
legend style={at={(-1.5,3.5)},font=\footnotesize}] % <= to define position and font legend
% the following are the "images" and numbers in the legend
    \addlegendimage{-stealth,red}
    \addlegendimage{stealth-stealth,dotted,green}
    \addlegendimage{stealth-stealth,dotted,black}
    \addlegendimage{stealth-stealth,dotted,blue}
    
\end{customlegend}

\end{tikzpicture}

\caption{M\'odulo \textbf{IGP}}
\label{fig:IGP}
\end{figure}

Adem\'as este sistema es el menor(en el sentido de n\'umero de especies) en el cual podemos encontrar 2 \emph{caminos de ensamblaje}.
\begin{figure}[h]
\begin{center}
\begin{tikzpicture}[->,>=stealth',shorten >=1pt,auto,
  thick,main node/.style={circle,fill=blue!20,draw,font=\sffamily\Large\bfseries}]

\node[main node](R1) at (1,0) { R};
\node[main node](C1) at (0,2) {C};
\node[main node](C2) at  (4,2){C};
\node[main node](R2) at ($(R1) + (3,0)$) {R};
\node[main node](P1) at (5,4){P};
\node[main node](R3) at ($(R2) + (3,0)$){R};
\node[main node](C3) at ($(R3) + (1,2)$){C};
\node[main node](P3) at ($(R3) + (0,4)$){P};
\node[font = \fontsize{15}{15}\selectfont](A1) at (-4,0) {Camino 1};

\path
(C1) edge[bend left,dotted] (R1)
($(R1)+(1.0,0)$) edge[line width = 2]  ($(R1)+ (2,0)$)
(R2) edge (C2)
(P1) edge[bend right,dotted] (C2)
(P1) edge[bend left,dotted] (R2)
($(R2) + (1.0,0)$) edge[line width  = 2] ($(R2) + (2,0)$)
(R3) edge (C3)
(R3) edge (P3)
(C3) edge (P3);
\end{tikzpicture}
$\left. \right.$\\
$\left. \right.$\\

\begin{tikzpicture}[->,>=stealth',shorten >=1pt,auto,
  thick,main node/.style={circle,fill=blue!20,draw,font=\sffamily\Large\bfseries}]

\node[main node](R1) at (1,0) { R};
\node[main node](P1) at (0,2) {P};
\node[main node](P2) at  (4,4){P};
\node[main node](R2) at ($(R1) + (3,0)$) {R};
\node[main node](C1) at (5,2){C};
\node[main node](R3) at ($(R2) + (3,0)$){R};
\node[main node](C3) at ($(R3) + (1,2)$){C};
\node[main node](P3) at ($(R3) + (0,4)$){P};
\node[font = \fontsize{15}{15}\selectfont](A1) at (-4,0) {Camino 2};

\path
(P1) edge[bend left,dotted] (R1)
($(R1)+(1.0,0)$) edge[line width = 2]  ($(R1)+ (2,0)$)
(R2) edge (P2)
(C1) edge[bend left,dotted] (R2)
(P2) edge[bend left,dotted,red] (C1)
($(R2) + (1.0,0)$) edge[line width  = 2] ($(R2) + (2,0)$)
(R3) edge (C3)
(R3) edge (P3)
(C3) edge (P3);

\end{tikzpicture}
\end{center}

\caption{\emph{Caminos de ensamblaje} para el m\'odulo \textbf{IGP}}
\label{fig:IGPAssembly}
\end{figure}
Numerosos ejemplos son citados en \cite{polis1989ecology}, algunos de los cuales se presentan en la siguiente tabla:
\improvement{HACER LA TABLA!}


\subsection{Cadenas Tr\'oficas}


