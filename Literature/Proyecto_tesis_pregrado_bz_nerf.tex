\documentclass[a4paper,12pt]{article}
\usepackage{amsmath}
\usepackage{bm}
\usepackage{latexsym}
\usepackage[left=3.5cm,right=2.5cm]{geometry}
\usepackage{diagbox}
\usepackage[table]{xcolor}
\usepackage{fontspec}
\setmainfont{Times New Roman}
\setlength{\parskip}{.1in}  
\setlength{\parindent}{0.0in}  
\renewcommand{\baselinestretch}{1.5}
\renewcommand\contentsname{Contenido}

\newcommand{\PP}{\textit{\tiny P}}
\newcommand{\C}{\textit{\tiny C}}
\newcommand{\R}{\textit{\tiny R}}
\newcommand{\CR}{\textit{\tiny CR}}
\newcommand{\PC}{\textit{\tiny PC}}
\newcommand{\PR}{\textit{\tiny PR}}
\newcommand{\CP}{\textit{\tiny CP}}
\newcommand{\RC}{\textit{\tiny RC}}
\newcommand{\RP}{\textit{\tiny RP}}


\begin{document}
\tableofcontents

\section{PLANTEAMIENTO DEL PROBLEMA,\\ DELIMITACI\'ON Y FORMULACI\'ON}
La longitud de las cadenas tr\'oficas es una propiedad estructural importante de los ecosistemas
,que afecta distintas funciones ecol\'ogicas de \'este (e.g. reciclaje de nutrienes, patr\'on de efectos cascada), por lo cual  la observaci\'on de la variaci\'on de su longitud despert\'o y sigue despertando el inter\'es de muchos investigadores, los cuales han tratado de determinar los factores que determinan esta longitud; si bien se han propuesto factores ambientales como posibles determinantes de la longitud: R\'egimen de disturbancias, Tama\~no del ecosistema y Nivel de productividad (Post 2002, Takimoto et al. 2012); a\'un no se ha llegado a una conclusi\'on definitiva sobre los mecanismos que regulan esta longitud( Sterner et al. 1997, Takimoto y Post 2012). En los \'ultimos a\~nos se a pasado de determinar qu\'e factor es m\'as importante a determinar c\'omo los distintos factores interactuan entre s\'i y bajo que contexto son m\'as o menos importantes(Post 2002).\\
La distribuci\'on de las razones de masas de la presa con respecto al depredador de las interacciones tr\'oficas presentes en una comunidad(Brose et al. 2006) ha sido relacionada con niveles de estabilidad din\'amica(Yodzis e Innes 1992,McCann y Yodzis 1995 , Brose et al. 2006,Pawar et al 2012), rangos de coexistencia(Pawar et al. 2012) y dada su estrecha relaci\'on con la distribuci\'on de masas de las especies presentes en el sistema,tasa de respiraci\'on del ecosistema(Brown et al. 2005). Sin embargo en conocimiento del autor no existen trabajos te\'oricos que hayan relacionado esta importante propiedad de las comunidades con la longitud de las cadenas tr\'oficas presentes en ellas(Emp\'iricamente Jennings y Warr 2003 sugieren la existencia de una relaci\'on).
Lo que se propone en este trabajo es evaluar las limitaciones que impondr\'ian la presencian de un conjunto particular de valores de la raz\'on de masas depredador-presa de una comunidad sobre la longitud de las cadenas tr\'oficas presentes en ella.


\section{JUSTIFICACI\'ON DE LA INVESTIGACI\'ON}
Conocer los mecanismos que regulan la longitud de las cadenas tr\'oficas(FCL de acuerdo a sus siglas en ingl\'es) y entender como estos operan , es de suma importancia tanto desde el punto de vista b\'asico como aplicado, ya que esta propiedad de los ecosistemas afecta distintas funciones de \'este, tales como el reciclaje de nutrientes(DeAngelis 1992), regulaci\'on tr\'ofica y patrones de cascadas tr\'oficas(Hairston et al. 1960, Oksanen y Oksanen 2000, Carpenter y Kitchell 1993) y la biomagnificaci\'on de toxinas( Cabana y Rasmussen 1994), al develar los mecanismos que finalmente controlan esta propiedad estructural de los ecosistemas, se ganar\'ia un mejor entendimiento de como opera la naturaleza y a su vez predecir el cambio que provocar\'ian ciertas actividades antropog\'enicas sobe estos.

\section{FORMULACI\'ON DE OBJETIVOS GENERALES Y ESPEC\'IFICOS}
\subsection{Objetivo General}
Desarrollar un mejor entendimiento de los mecanismos que regulan la longitud de las cadenas tr\'oficas.
\subsection{Objetivos Espec\'ificos}
\begin{itemize}
\item Derivar condiciones necesarias y suficientes para la expresi\'on de los mecanismos responsables de la variaci\'on en la longitud de las cadenas tr\'oficas(e.g. inserci\'on, adici\'on y  grado omnivor\'ismo); dependientes de la raz\'on de masas depredador-presa presentes en la comunidad.
\item Evaluar la influencia de factores como la dimensi\'on del ecosistema, la estrategia de forrajeo de los depredadores presentes y el nivel de productividad basal del ambiente, sobre dichas condiciones.
\end{itemize}
\section{HIP\'OTESIS Y VARIABLES DE LA INVESTIGACI\'ON}
\subsection{Hip\'otesis}
\begin{itemize}
\item[$H_{o1}:$] La longitud de las cadenas tr\'oficas es invariante respecto a cambios en la distribuci\'on de raz\'on de masas de depredador-presa presentes en la comunidad.
\item[$H_{11}:$] La longitud de las cadenas tr\'oficas es dependiente de la distribuci\'on de raz\'on de masas de depredador-presa presentes, debido a las limitaciones que estos imponen sobre los mecanismos de inserci\'on, adici\'on y omnivorismo.
\item[$H_{o2}:$] Las limitaciones impuestas sobre la longitud de las cadenas tr\'oficas por la distribuci\'on de raz\'on de masas de depredador-presa presentes en la comunidad es independiente de la dimensi\'on del ecosistema, la estrategia de forrajeo de los depredadores presentes y el nivel de productividad basal del ambiente.
\item[$H_{12}:$] La dimensi\'on del ecosistema, la estrategia de forrajeo de los depredadores presentes y el nivel de productividad basal del ambiente influencian las limitaciones que impone sobre la longitud de las cadenas tr\'oficas la distribuci\'on de raz\'on de masas de depredador-presa presentes en la comunidad.
\end{itemize}
\subsection{Variables}
\subsubsection{Independientes}
\begin{itemize}
\item Dimensi\'on del Ecosistema
\item Estrategia de Forrajeo
\item Nivel de Productividad basal.
\item Razon de masas de depredador presa($m_R/m_C$), donde $m_R$ es la masa de la presa y $m_C$ es la masa del depredador.
\item Masa del depredador tope.
\end{itemize}
\subsubsection{Dependientes}
\begin{itemize}
\item Longitud de cadena tr\'ofica(FCL).
\item Condiciones necesarias y suficientes para la expresi\'on del mecanismo de \textit{adici\'on}
\item Condiciones necesarias y suficientes para la expresi\'on del mecanismo de \textit{inserci\'on}
\item Grado de Omnivorismo

\end{itemize}
\section{ANTECEDENTES}
La b\'usqueda de los determinantes de la longitud de las cadenas tr\'oficas se remonta a la pregunta hecha por Elton(1927) : ¿`Qu\'e limita la longitud de las cadenas tr\'oficas?, esto debido a que emp\'iricamente se estaba encontrando cadenas no mayores de 5 pasos, numerosos estudios se han conducido desde entonces, emergiendo diversas hip\'otesis, las cuales ser\'an citadas a continuaci\'on acompa\~nadas de sus respectivos proponentes.\\
\begin{enumerate}
\item \textbf{Limitaciones energ\'eticas}: Propuesta por Lindeman(1942), extendida por Hutchinson(1959) y por Schoener(1989), este \'ultimo consider\'o al espacio expl\'icitamente. Siendo esto una simple adaptaci\'on de la segunda ley de la termodin\'amica y en la cual se propones que la longitud est\'a limitada debido a que la transferencia de energ\'ia no es perfecta de manera tal que a trav\'es de cada nivel tr\'ofico se va perdiendo parte de la energ\'ia producida por el nivel tr\'ofico previo, por lo que conforme aumenta el n\'umero de niveles tr\'oficos la energ\'ia disponible hacia el nivel tr\'ofico superior(si existiera) es cada vez menor, no permitiendo la existencia de muchos niveles tr\'oficos.
\item \textbf{Estabilidad din\'amica}: Postulada por Pimm y Lawton(1977), donde te\'oricamente demostraron que una cadena m\'as larga era m\'as inestable( medida de estabilidad de acuerdo al tiempo de vuelta hacia el estado de equilibrio), sin embargo Sterner et al.(1997) han refutado esta idea, haciendo ver que lo que supuestamente encontraron Pimm y Lawton era dependiente de los factores de regulaci\'on presentes en el sistema(e.g. denso-dependencia intraespec\'ifica), lo cual Pimm y Lawton incorporaron solo a nivel basal, si se controlan los factores a medida que uno alarga la cadena, en contradicci\'on con lo originalmente postulado, las cadenas se vuelven m\'as estables(Sterner et al. 1997).
\item \textbf{Tama\~no del ecosistema}: Postulada entre otros por Cohen y Newman(1991)y Post et al.(2000), donde sugieren que el tama\~no del ecosistema es el responsable de la longitud observada en las cadenas tr\'oficas, relacionada con las ideas de Curva especie-\'area,i.e mayor \'area implica mayor n\'umero de especies lo que potencialmente da lugar a una cadena tr\'ofica m\'as larga (Holt et al. 1999, Holt 2002), como tambi\'en sugiriendo el hecho que una mayor \'area aminora las inestabilidades din\'amicas  que puede tener un sistema limitado espacialmente(McCann et al. 2004), esta hip\'otesis est\'a obteniendo respaldo emp\'irico(Post et al. 2000, Takimoto et al. 2007, Takimoto y Post 2012).
\item \textbf{Din\'amica adaptativa} Un trabajo reciende de Kondoh y Ninomiya(2009) rejuvenecen la idea de Hastings y Conrad(1979) proponiendo que la din\'amica adaptativa del depredador al momento de elegir a su presa es un factor clave en la limitaci\'on de la longitud de las cadenas tr\'oficas.
\end{enumerate}
Como se ve de lo arriba mostrado, a pesar de la gran cantidad de trabajo realizado en este tema, no hay a\'un una conclusi\'on definitva, debido a lo cual Takimoto y Post(2012) sugieren que, con la teor\'ia actual a\'un no se pueden entender algunos patrones observados en la naturaleza.
\section{METODOLOG\'IA}
\subsection{Modelaci\'on matem\'atica}
Se har\'a uso de un modelo de 3 especies cuya estructura de interacci\'on ser\'a  el modulo IGP(Intraguild predation, Polis et al.1989, Holt y Polis 1997) para describir las interacciones entre las 3 especies componentes(Recurso, consumidor intermedio(IG prey) y depredador tope(IG Predator) );debido a que ha sido sugerido como un sistema de complejidad suficiente para incorporar distintos mecanimos que provocan variaci\'on en la longitud de las cadenas tr\'oficas(e.g. Inserci\'on, Adici\'on , Omnivorismo;Takimoto y Post 2007).
\subsubsection*{Forma general}
La din\'amica local para cada parche $i$ estar\'a gobernada ,de forma general,por el siguiente sistema de ecuaciones diferenciales:
\begin{align}
&\frac{dR}{dt} = F(R) -G(R,C)-H_{\R}(R,C,P)  \\
&\frac{dC}{dt} = \beta G(R,C)-H_{\C}(R,C,P) - q_{\C}C  \\
&\frac{dP}{dt} = \delta H_{\R}(R,C,P) +\gamma H_{\C}(R,C,P) -q_{\textit{\tiny P}}P
\end{align}
\subsubsection*{Donde:}
\begin{itemize}
\item[]$R$: Densidad de biomasa del recurso $R$.
\item[]$C$: Densidad de biomasa del consumidor intermedio(IG prey) $C$.
\item[]$P$: Densidad de biomasa del depredador tope(IG predator) $P$.
\item[]$F$: Funci\'on que describe la din\'amica poblacional del recurso $R$ en ausencia de depredadores.
\item[]$G$: Funci\'on que describe la depredaci\'on ejercida por el consumidor $C$ sobre el recurso $R$.
\item[]$H_{\R}$: Funci\'on que describe la depredaci\'on ejercida por el depredador tope $P$ sobre el recurso $R$.
\item[]$H_{\C}$: Funci\'on que describe la depredaci\'on ejercida por el depredador tope $P$ sobre el consumidor $C$.
\item[]$\beta$: Eficiencia de conversi\'on de biomasa del recurso $R$ en biomasa del consumidor intermedio $C$.
\item[]$\delta$: Eficiencia de conversi\'on de biomasa del recurso $R$ en biomasa del depredador tope $P$.
\item[]$\gamma$: Eficiencia de conversi\'on de biomasa del consumidor $C$ en biomasa del depredador tope $P$.
\item[]$q_{\C}$: Tasa de perdida de biomasa del consumidor intermedio $C$.
\item[]$q_{\textit{\tiny P}}$: Tasa de perdida de biomasa del depredador tope $P$.
\end{itemize}
\subsubsection*{Forma espec\'ifica}
Para este caso se usar\'a la forma de crecimiento log\'istico(Gotelli 2001) para $f$ y respuestas funcionales tipo I para $G$ y $H$ (Holling 1959), este sistema de ecuaciones es una modificaci\'on de la formulaci\'on de Lotka y Volterra.
\begin{align}
&\frac{dR}{dt} = R \left[ r(1-\frac{R}{K}) -(\frac{\alpha_{\CR}}{m_\C})C-(\frac{\alpha_{\PR}}{m_\textit{\tiny P}})P \right] &\\
&\frac{dC}{dt} = C \left[ \beta(\frac{\alpha_{\CR}}{m_\C})R-(\frac{\alpha_{\PC}}{m_\PP})P -q_{\C}  \right]\\
&\frac{dP}{dt} = P \left[ \delta (\frac{\alpha_{\PR}}{m_\PP})R  +\gamma (\frac{\alpha_{\PC}}{m_\PP})C-q_{\PP} \right]
\end{align}
\subsubsection*{Donde:}
\begin{itemize}
\item[]$r$: Tasa intr\'inseca de producci\'on de biomasa del Recurso $R$.
\item[]$K$: Capacidad de carga (en biomasa) del recurso $R$.
\item[]$\alpha_{\textit{\tiny QJ}} $: Tasa per c\'apita de b\'usqueda de biomasa de la especie $J$  por parte del consumidor $Q$ en el parche.$Q=C,P$ y $J=R,C$.
\item[]$m_\R$: masa de un individuo ''t\'ipico'' del recurso $R$.
\item[]$m_\C$: masa de un individuo ''t\'ipico'' del consumidor intermedio $C$.
\item[]$m_\textit{\tiny P}$: masa de un individuo ''t\'ipico'' del depredador tope $P$.
\end{itemize}
Los par\'ametros y funciones no mencionados mantienen la descripci\'on dada en el caso general.\\

\subsubsection*{Parametrizaci\'on}
Se har\'a uso de argumentos bioenerg\'eticos y biomec\'anicos para designar a priori el valor de cierto par\'ametros del modelo, lo cual disminuir\'a el espacio param\'etrico del sistema y nos permitir\'a enfocarnos en los para\'ametros deseados(Yodzis e Innes 1992,Brown et al. 2004,Savage et al. 2004,Brose et al. 2006, Pawar et al. 2012). Los par\'ametros se especifican a continuaci\'on
\begin{itemize}
\item[]$r = r_0m_\R^{-0.25} $
\item[]$q_{C}=q_{0,C}m_\C^{-0.25}$
\item[]$q_{P}=q_{0,P}m_\PP^{-0.25}$
\item[]$K= K_{0}m_\R^{0.25} $
\item[]$\alpha_{\CR}= \alpha_{0,C}m_\C^{p_v+2p_d(D_R-1)}f(k_{\RC})$
\item[]$\alpha_{\PR}= \alpha_{0,P}m_\PP^{p_v+2p_d(D_R-1)}f(k_{\RP})$
\item[]$\alpha_{\PC}= \alpha_{0,P} m_P^{p_v+2p_d(D_C-1)}f(k_{\CP})$
\item[]$t_{h_{\RC}}=t_{h,0}m_\C^{-0.75}$
\item[]$t_{h_{\CP}} =t_{h_{\RP}}=t_{h,0}m_\PP^{-0.75}$
\item[]$D_J$ : Dimensi\'on del habitat donde habita la especie $J$ . $J = R,C$.
\item[]$p_v$ : Exponente que controla como escala la velocidad de un individuo con el tamaño corporal; su valor depende de la dimensi\'on del habitat.
\item[]$p_d$ : Exponente que controla como escala la distancia de reacci\'on de un depredador con el tamaño corporal; su valor depende de la dimensi\'on del habitat.
\item[]$ k_{\textit{\tiny QJ}}$ : Raz\'on de masas de la presa $Q$ respecto al depredador $J$ . $Q = R,C$ y $J= C,P$.	
\end{itemize}
La forma de la funci\'on $f$ depende del tipo de estrategia usado por el depredador:
\begin{itemize}
\item[] Pastoreo : \[ f(k) = k^{p_v+p_d(D - 1)}  \]  
\item[] Captura Activa : \[ f(k) = \sqrt {1+  k ^{2 p_v}} k^{p_d(D - 1)} \]
\item[] Captura pasiva( \textit{ Sit and wait} sensu Pawar et al. 2012) : \[ f(k) = k^ {p_d(D - 1)} \]
\end{itemize}
Como se observa el valor de los par\'ametros presenta una relaci\'on potencial con la masa de las especies implicadas ,los argumentos usados para derivar esta relaci\'on estan descritas en Savage et al.2004 y Pawar et al. 2012(Informaci\'on suplementaria) . El valor de las constantes de normalizaci\'on depende de diversos factores, se usar\'an valores previamente determinados emp\'iricamente(Peters 1983,Yodzis e Innes 1992,Brown et al. 2004,Savage et al. 2004,Brose et al. 2006, Pawar et al. 2012), excepto para $K_0$ que denota el nivel de productividad basal del habitat para el cual se usar\'an los siguientes valores : $0.001,0.01,0.1,1$ para $D =2 $ (e.g. ambientes bent\'onicos o terrestres ) y $1,10,100,150$ y $300$ para $D=3$.(e.g. zona pel\'agica marina).

\subsection{An\'alisis}
Todos los an\'alisis descritos a continuaci\'on se desarrollar\'an para distintas combinaciones de dimensi\'on del habitat, estrategia de forrajeo y nivel de productividad ambiental basal.
\subsubsection{Criterios de Invasibilidad}
Dada la parametrizaci\'on empleada es posbile derivar criterios de invasibilidad en t\'erminos de las razones de masa de depredador-presa presentes en el m\'odulo IGP y la masa del depredador tope; se analizar\'an los siguientes escenarios:
\begin{itemize}
\item $C$ invade un sistema conformado solo por $R$.
\item $P$ invade un sistema conformado solo por $R$.
\item $P$ invade un sistema conformado por $R$ y $C$.
\item $C$ invade un sistema conformado por $R$ y $P$.
\end{itemize}
En los 3 primeros escenarios la variaci\'on de la Longitud de la cadena tr\'ofica involucra al mecanismo de \textit{adici\'on} y en el \'ultimo escenario el mecanismo involucrado es el de \textit{inserci\'on}.
\subsubsection{M\'axima longitud}
Dentro de la regi\'on de coexistencia de las tres especies(i.e. $(R^*,C^*,P^*) \in \mathbf{R}^3_+$) , se determinar\'a que valores de la raz\'on de masas de depredador-presa y masa de depredador tope implican una longitud de cadena tr\'ofica m\'axima, es decir que valores corresponden con el m\'inimo grado de omnivorismo del sistema.\\
La M\'axima posici\'on tr\'ofica(i.e. posici\'on tr\'ofica del depredador tope) ser\'a medida de la siguiente manera(Takimoto y Post 2007):
\begin{equation} TP_j = \sum_{i \in G}^n (1+TP_i) p_i  \end{equation} donde $G$ es el cojunto de todas las presas del depredador $J$ ,$TP_i$ es la posici\'on tr\'ofica de la especie $i$ y $p_i$ es la proporci\'on de la biomasa total ingerida por el depredador $j$ que deriva de la especie $i$ tal que $ \sum_{i \in G} p_i =1 $. \\
En nuestro caso esto se particulariza a :
\begin{equation} \mathbf{TP}= \frac{2 I_R + 3 I_C}{I_R+I_C} \end{equation}
Donde:
\begin{itemize}
\item $I_R$ = biomasa de Recurso basal ingerida.
\item $I_C$ = biomasa de IGprey ingerida.
\end{itemize}
El grado de omnivor\'ismo $\mathbf{O}$ :
\begin{equation} \mathbf{O}= \frac{I_R}{I_R+I_C} \end{equation}
\subsubsection{Estabilidad din\'amica}
Dentro de la zona de coexistencia se categorizar\'an zonas como estables o inestables siguiendo el criterio de Routh-Hurwtiz (May 2001).

\section{Cronograma de Desarrollo}
\begin{center}
\begin{tabular}{|c|c|c|c|c|}
\hline
A\~no & \multicolumn{4}{c|}{2014}\\
\hline
\diagbox{Actividad}{mes}& Marzo&Abril&Mayo&Junio \\
\hline
Recopilaci\'on de literatura & \cellcolor[gray]{0.9} & & & \\
\hline
An\'alisis matem\'atico & & \cellcolor[gray]{0.9} & & \\
\hline
An\'alisis de los resultados obtenidos & & & \cellcolor[gray]{0.9} & \\
\hline
Redacci\'on de la tesis & & & & \cellcolor[gray]{0.9} \\
\hline
\end{tabular}
\end{center}
\section{Presupuesto global}
\begin{center}
\begin{tabular}{|p{3.5cm}|c|c|c|c|}
\hline
& Medida & Cantidad & Precio Unitario(S \textbackslash .) & Precio Total (S \textbackslash .) \\
\hline
\multicolumn{5}{|l|}{\cellcolor[gray]{0.9} Bienes}\\
\hline
\multicolumn{5}{|l|}{Libros}\\
\hline
Matrices and Graphs Stability Problems in Mathematical Ecology. Dmitrii Logofet, CRC Press 1993. & Unidad & 1 & 112 & 112 \\
\hline
\multicolumn{5}{|l|}{Materiales}\\
\hline
Blocks & Unidad & 5 & 4 & 20 \\
\hline
Lapiceros & Unidad & 5 & 2 & 10 \\
\hline
L\'apices & Unidad & 10 & 1 & 10\\ 
\hline
Borrador & Unidad & 5 & 1 & 5 \\
\hline
Tajador & Unidad & 2 & 2 & 4\\
\hline
Laptop Lenovo Ideapad U410 & Unidad & 1  & 3100 & 3100 \\ 
\hline
\multicolumn{5}{|l|}{\cellcolor[gray]{0.9} Servicios}\\
\hline
Impresiones & Unidad & 800 & 0.05 & 40 \\ 
\hline
Honorarios & D\'ia & 30 & 80 & 2400 \\
\hline
\multicolumn{4}{|l|}{Total} & 5701 \\ 
\hline
\end{tabular}	

\end{center}

\section{REFERENCIAS BIBLIOGRAFICAS}
\begin{itemize}
\item[] Brose, U., R. J. Williams, y N. D. Martinez. 2006. Allometric scaling enhances stability in complex food webs. Ecology Letters 9:1228–1236
\item[] Brose, U., T. Jonsson, E. L. Berlow, P. Warren, C. Banasek-Richter, L.-F. Bersier, J. L. Blanchard, T. Brey, S. R. Carpenter, y M.-F. C. Blandenier. 2006. Consumer-resource body-size relationships in natural food webs. Ecology 87:2411–2417.\\
\item[] Brown, J. H., J. F. Gillooly, A. P. Allen, V. M. Savage, y G. B. West. 2004. Toward a metabolic theory of ecology. Ecology 85:1771–1789. \\
\item[] Cabana,G. y J.B., Rasmussen.1994. Modelling food chain structure and contaminant bioacumulation using stable nitrogen isotopes.Nature 372:255-257 \\
\item[] Calcagno,V.;Massol,F.,Mouquet,N.,Jarne,P. y  P. David.2011.Constraints on food chain length arising from regional metacommunity dynamics. Proc. R. Soc. B 278(1721):3042-3049\\
\item[] Carpenter,S.R y J.F. Kitchell.1993. The trophic cascade in lake ecosystems. Cambridge, UK:Cambridge Universisty Press.\\
\item[] Chesson, D. 2000. Mechanism of maintenance of species diversity. Ann. Rev. Ecol. Syst. 31:343-366\\
\item[] Chesson, J. 1983. The Estimation and Analysis  of Preference and Its relationship with Foraging models.Ecology 64(5):1297-1304\\
\item[] Cohen JE, Newman CM (1991) Community area and food-chain length: theoretical predictions. Am Nat 138:1542–1554\\
\item[] Elton, C. 1927. Animal Ecology.  Sidgwick y Jackson, Londres\\
\item[] Hairsont,N.G., Smith,F.E y L.B. Slobdokin.1960. Community structure, population control and competition. Am. Nat.44,421-425\\
\item[] Hastings,H.M. y M,Conrad.1979. Length and evolutionary stability on fodd chains. Nature 282,838-839.\\
\item[] Holt, R.D .1993. Ecology at the mesoscale: the influence of regional processes on local communities. Species diversity in ecological communities, Ricklefs RE, Schluter D. University of Chicago Press, Chicago, pp 77–88\\
\item[] Holt,R.D 1984. Spatial Heterogeinity, Indirect interactions and the coexistence of prey species. Am.Nat. 124:377-406\\
\item[] Holt, R. D. 1997 From metapopulation dynamics to community structure: some consequences of spatial heterogeneity. In Metapopulation biology: ecology, genetics, and evolution (eds I. A. Hanski y M. E. Gilpin), pp. 149–164. San Diego, CA: Academic Press.\\
\item[] Holt,R.D.y G. Polis. 1997. A Theoretical framework for intraguild predation. Am. nat. 149:745-764\\
\item[] Holt,R.D.;Lawton,J.H.;Polis,G.A. y N.D. Martinez. 1999. Trophic rank and the species-area relationship. Ecology 80(5):1495-1504\\
\item[] Holt, R.D.2002. Food webs in space: On the interplay of dynamic stability and spatial processes. Ecological Research 17:261-273\\
\item[] Hutchinson, G. E. 1959 Homage to Santa Rosalia, or why are there so many kinds of animals?. Am. Nat. 93, 145–159\\
\item[] Kondoh, M. y Ninomiya, K. 2009 Food-chain length and adaptive foraging. Proc. R. Soc. B 276, 3113–3121.\\
\item[] Lindeman, RL. 1942. The trophic-dynamic aspect of ecology. Ecology 23: 399–417\\
\item[] May, R . 2001. The stability of Model Ecosystems.Princeton University Press.\\
\item[] McCann,K.S.;Rasmussen,J.B. y J. Umbanhowar.2005. The dynamics of spatially coupled food webs. Ecology letter 8: 513-523\\
\item[] McCann, K., y P. Yodzis. 1995. Bifurcation structure of a three-species food-chain model. Theoretical population biology 48:93–125. \\ 
\item[] McCann,K.S.; Hastings,A. y G.A. Huxel .1998. Weak trophic interaction and the balance of nature.Nature 395:794-797\\
\item[] Mouquet,N.; Hoppes, M.F y P, Amarasekare. 2006. The world is patchy and Heterogeneous!Trade-off and source-sink Dynamics in competitive Metacommunities. In Leibold et.al. 2006. Metacommunities, Spatial dynamics and Ecological communities. Chicago press.\\
\item[] Oksanen, L.; Fretwell,S.D.; Arruda, J. y P. Niemela¨.1981. Exploitation ecosystems in gradients of primary productivity. Am Nat 118:240–261\\
\item[] Peters, R. H. 1986. The ecological implications of body size. Cambridge University Press\\
\item[] Polis ,G.A. ; Myers, C.A. y R.D. Holt.1989. The Ecology and Evolution of Intraguild Predation: Potential competitors that eat each other.Annu. Rev.Ecol.Syst. 20:297-330\\
\item[] Post,D.M. .2002. The long and short of Food chain length. Trends in Ecology and Evolution.\\
\item[] Post,D.M .2007. Testing the productive-space hypothesis:rational and power.Oecologia 153:973-984.\\
\item[] Post, D. y G. Takimoto. 2007 Proximate structural mechanisms for variation in food-chain length. Oikos 116, 775–782\\
\item[] Pimm SL, Lawton JH .1977.The number of trophic levels in ecological communities. Nature 275:542–544\\
\item[] Pimm SL (1982) Food webs. Chapman and Hall, London \\
\item[] Savage,V.M.;Gillooly J.F ;Brown,J.H; West, G.B y E. L. Charnov 2004. Effects of Body size and Temperature on population growth. Am. Nat.163, 429-441.\\
\item[] Schoener, T. W. 1989 Food webs from the small to the large. Ecology 70, 1559–1589\\
\item[] Sternert,R.W.;Bajpai,A. y A. Thomas. 1997. The enigma of Food Chin Length: Absence of theoretical evidence for dynamic constraints. Ecology 1997:2258-2262\\
\item[] Takimoto, G.;Post,D.;Spiller,D.A. y R.D. Holt.2012. Effects of productivity, disturbance and ecosystem size on food-chain length: insights from a metacommunity model of intraguild predation.Ecol Res(2012) 27:481-493\\
\item[] Takimoto,G.;Spiller,D.A. y D.M. Post.2008. Ecosystem size, but not disturbance, determines food chain length on islands of the Bahamas. Ecology 2008:3001-3007\\
\item[] Takimoto, G. y D.M. Post. 2012. Environmental determinants of food-chain length: a meta analysis.Ecol Res. Published online 26 April 2012.\\
\item[][] Yodzis,P y S. Innes. 1992. Body size and consumer resource dynamics. Am. Natu. 139:1151-1175\\
\end{itemize}
\end{document}
